\chapter{Vector Spaces}

\section{Vector Spaces}

\Exercise1 If $F$ is a field, verify that $F^n$ (as defined in
Example~1) is a vector space over the field $F$.
\begin{proof}
  We need to check that addition and scalar multiplication, as defined
  in Example~1, satisfy conditions (3) and (4) of the definition. Let
  \begin{align*}
    \alpha &= (\alpha_1,\alpha_2,\dots,\alpha_n), \\
    \beta &= (\beta_1,\beta_2,\dots,\beta_n), \\
    \intertext{and}
    \gamma &= (\gamma_1,\gamma_2,\dots,\gamma_n)
  \end{align*}
  be arbitrary vectors in $F^n$. From the commutativity of addition in
  $F$, we have
  \begin{equation*}
    \alpha + \beta
    = (\alpha_1 + \beta_1, \dots, \alpha_n + \beta_n)
    = (\beta_1 + \alpha_1, \dots, \beta_n + \alpha_n)
    = \beta + \alpha,
  \end{equation*}
  so addition is commutative in $F^n$. Similarly, by associativity of
  addition in $F$, we have
  \begin{align*}
    \alpha + (\beta + \gamma)
    &= (\alpha_1 + (\beta_1 + \gamma_1),
      \dots, \alpha_n + (\beta_n + \gamma_n)) \\
    &= ((\alpha_1 + \beta_1) + \gamma_1,
      \dots, (\alpha_n + \beta_n) + \gamma_n) \\
    &= (\alpha + \beta) + \gamma,
  \end{align*}
  and associativity holds in $F^n$. The unique $0$ vector is
  \begin{equation*}
    0 = (0, 0, \dots, 0),
  \end{equation*}
  and it is clear that $\alpha + 0 = \alpha$. The unique additive
  inverse of $\alpha$ is given by
  \begin{equation*}
    -\alpha = (-\alpha_1, -\alpha_2, \dots, -\alpha_n),
  \end{equation*}
  and certainly $\alpha + (-\alpha) = 0$. The conditions in (3) are
  satisfied.

  Now let $c$ and $d$ be scalars in $F$. Then
  \begin{gather*}
    1\alpha
    = (1\alpha_1,\dots,1\alpha_n)
    = (\alpha_1,\dots,\alpha_n) = \alpha, \\
    (cd)\alpha
    = ((cd)\alpha_1, \dots, (cd)\alpha_n)
    = (c(d\alpha_1), \dots, c(d\alpha_n))
    = c(d\alpha), \\
    \begin{split}
      c(\alpha + \beta)
      &= (c(\alpha_1 + \beta_1),\dots,c(\alpha_n + \beta_n)) \\
      &= (c\alpha_1 + c\beta_1,\dots,c\alpha_n + c\beta_n) \\
      &= c\alpha + c\beta,
    \end{split} \\
    \intertext{and}
    \begin{split}
      (c + d)\alpha
      &= ((c + d)\alpha_1,\dots,(c+d)\alpha_n) \\
      &= (c\alpha_1 + d\alpha_1,\dots, c\alpha_n + d\alpha_n) \\
      &= c\alpha + d\alpha,
    \end{split}
  \end{gather*}
  so the conditions in (4) are satisfied. Therefore $F^n$ is a vector
  space over $F$.
\end{proof}

\Exercise2 If $V$ is a vector space over the field $F$, verify that
\begin{equation*}
  (\alpha_1 + \alpha_2) + (\alpha_3 + \alpha_4)
  = [\alpha_2 + (\alpha_3 + \alpha_1)] + \alpha_4
\end{equation*}
for all vectors $\alpha_1$, $\alpha_2$, $\alpha_3$, and $\alpha_4$ in
$V$.
\begin{proof}
  We only need to make use of commutativity and associativity of
  vector addition:
  \begin{align*}
    (\alpha_1 + \alpha_2) + (\alpha_3 + \alpha_4)
    &= (\alpha_2 + \alpha_1) + (\alpha_3 + \alpha_4) \\
    &= \alpha_2 + [\alpha_1 + (\alpha_3 + \alpha_4)] \\
    &= \alpha_2 + [(\alpha_1 + \alpha_3) + \alpha_4] \\
    &= [\alpha_2 + (\alpha_1 + \alpha_3)] + \alpha_4 \\
    &= [\alpha_2 + (\alpha_3 + \alpha_1)] + \alpha_4. \qedhere
  \end{align*}
\end{proof}

\Exercise3 If $C$ is the field of complex numbers, which vectors in
$C^3$ are linear combinations of $(1,0,-1)$, $(0,1,1)$, and $(1,1,1)$?
\begin{solution}
  A vector $\alpha = (y_1,y_2,y_3)$ is a linear combination of
  $(1,0,-1)$, $(0,1,1)$, and $(1,1,1)$ if there are scalars
  $x_1,x_2,x_3$ such that
  \begin{equation*}
    x_1(1, 0, -1) + x_2(0, 1, 1) + x_3(1, 1, 1) = \alpha,
  \end{equation*}
  which leads to the following system of equations:
  \begin{alignat*}{4}
    \phantom{{}-}x_1 &&  &{}+{}& x_3 &{}={}& y_1 \\
    && x_2 &{}+{}& x_3 &{}={}& y_2 \\
    -x_1 &{}+{}& x_2 &{}+{}& x_3 &{}={}& y_3\rlap.
  \end{alignat*}
  Since the coefficient matrix
  \begin{equation*}
    A =
    \begin{bmatrix}
      1 & 0 & 1 \\
      0 & 1 & 1 \\
      -1 & 1 & 1
    \end{bmatrix}
  \end{equation*}
  is row-equivalent to the identity matrix, this system of equations
  has a solution for each $\alpha$. Therefore, all vectors in $C^3$
  are linear combinations of the vectors $(1,0,-1)$, $(0,1,1)$, and
  $(1,1,1)$.
\end{solution}

\Exercise4 Let $V$ be the set of all pairs $(x,y)$ of real numbers,
and let $F$ be the field of real numbers. Define
\begin{align*}
  (x,y) + (x_1,y_1) &= (x + x_1, y + y_1) \\
  c(x,y) &= (cx,y).
\end{align*}
Is $V$, with these operations, a vector space over the field of real
numbers?
\begin{solution}
  No, $V$ is not a vector space. Most of the conditions are satisfied,
  but distributivity over scalar addition fails when $y$ is nonzero:
  \begin{equation*}
    (c + d)(x,y) = ((c + d)x, y) = (cx + dx, y)
  \end{equation*}
  but
  \begin{equation*}
    c(x,y) + d(x,y) = (cx, y) + (dx,y) = (cx + dx, 2y). \qedhere
  \end{equation*}
\end{solution}

\Exercise5 On $R^n$, define two operations
\begin{align*}
  \alpha \oplus \beta &= \alpha - \beta \\
  c\cdot\alpha &= -c\alpha.
\end{align*}
The operations on the right are the usual ones. Which of the axioms
for a vector space are satisfied by $(R^n,\oplus,\cdot)$?
\begin{solution}
  Commutativity of $\oplus$ fails, since $\alpha - \beta$ is not, in
  general, equal to $\beta - \alpha$. Associativity of $\oplus$ also
  fails since, for nonzero $\gamma$,
  \begin{equation*}
    (\alpha - \beta) - \gamma
    \neq \alpha - \beta + \gamma
    = \alpha - (\beta - \gamma).
  \end{equation*}

  The usual zero vector still works, since
  $\alpha\oplus0 = \alpha - 0 = \alpha$. Additive inverses also exist,
  but they are not the usual ones. Instead, each vector is its own
  inverse, since $\alpha\oplus\alpha = \alpha - \alpha = 0$.

  For multiplication $\cdot$, it is not the case that
  $1\cdot\alpha = \alpha$ since, for nonzero $\alpha$,
  $1\cdot\alpha = -1\alpha \neq \alpha$. Associativity with scalar
  multiplication does not hold either, since
  \begin{equation*}
    (c_1c_2)\cdot\alpha
    = -c_1c_2\alpha
  \end{equation*}
  while
  \begin{equation*}
    c_1\cdot(c_2\cdot\alpha)
    = c_1\cdot(-c_2\alpha)
    = c_1c_2\alpha.
  \end{equation*}
  For the distributive properties, we have
  \begin{align*}
    c\cdot(\alpha\oplus\beta)
    &= c\cdot(\alpha - \beta) \\
    &= -c(\alpha - \beta) \\
    &= -c\alpha + c\beta
  \end{align*}
  and
  \begin{align*}
    c\cdot\alpha\oplus c\cdot\beta
    &= -c\alpha - (-c\beta) \\
    &= -c\alpha + c\beta,
  \end{align*}
  so the first distributive property holds. And
  \begin{align*}
    (c_1 + c_2)\cdot\alpha
    &= -(c_1 + c_2)\alpha \\
    &= -c_1\alpha - c_2\alpha,
  \end{align*}
  while
  \begin{align*}
    c_1\cdot\alpha \oplus c_2\cdot\alpha
    &= -c_1\alpha - (-c_2\alpha) \\
    &= -c_1\alpha + c_2\alpha,
  \end{align*}
  so the second distributive property fails.

  To summarize, in the vector space definition, only properties (c)
  and (d) of (3) and (c) of (4) hold.
\end{solution}

\Exercise6 Let $V$ be the set of all complex-valued functions $f$ on
the real line such that (for all $t$ in $R$)
\begin{equation*}
  f(-t) = \overline{f(t)}.
\end{equation*}
The bar denotes complex conjugation. Show that $V$, with the
operations
\begin{align*}
  (f + g)(t) &= f(t) + g(t) \\
  (cf)(t) &= cf(t)
\end{align*}
is a vector space over the field of {\em real} numbers. Give an
example of a function in $V$ which is not real-valued.
\begin{solution}
  Commutativity and associativity of addition follow from the
  properties of addition in $C$. Note that the zero function is in
  $V$. If $f\in V$, then the function $-f$ given by
  \begin{equation*}
    (-f)(t) = -f(t)
  \end{equation*}
  is in $V$ since
  \begin{equation*}
    -f(-t) = -\overline{f(t)}
    = \overline{-f(t)} = \overline{(-f)(t)}.
  \end{equation*}
  And $f + (-f)$ is the zero function.

  For scalar multiplication, we have
  \begin{equation*}
    (1f)(t) = f(t),
  \end{equation*}
  so the first property is satisfied. And
  \begin{equation*}
    ((cd)f)(t) = (cd)f(t) = c((df)(t)) = (c(df))(t),
  \end{equation*}
  so the second property is satisfied. And distributivity holds, since
  \begin{align*}
    (c(f + g))(t) &= c(f(t) + g(t)) \\
                  &= cf(t) + cg(t) \\
                  &= (cf)(t) + (cg)(t) \\
                  &= (cf + cg)(t)
  \end{align*}
  and
  \begin{align*}
    ((c + d)f)(t) &= (c + d)f(t) \\
                  &= cf(t) + df(t) \\
                  &= (cf)(t) + (df)(t) \\
                  &= (cf + df)(t).
  \end{align*}
  Therefore $V$ is a vector space over $R$.

  For an example of a function in $V$, consider the function $f$ from
  $R$ to $C$ given by
  \begin{equation*}
    f(t) = ti.
  \end{equation*}
  Then $f(-t) = -ti = \overline{ti} = \overline{f(t)}$ as required.
\end{solution}
