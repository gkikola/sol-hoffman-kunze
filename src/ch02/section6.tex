\skipsection
\section{Computations Concerning Subspaces}

\Exercise1 Let $s < n$ and $A$ an $s\times n$ matrix with entries in
the field $F$. Use Theorem~4 (not its proof) to show that there is a
non-zero $X$ in $F^{n\times1}$ such that $AX = 0$.
\begin{proof}
  Let $\alpha_1,\dots,\alpha_n$ denote the columns of $A$. Then each
  $\alpha_i$ is a member of the vector space $F^s$, which has
  dimension strictly less than $n$. Therefore, by Theorem~4, the
  $\alpha_i$ are necessarily linearly dependent. Thus we can write
  \begin{equation*}
    c_1\alpha_1 + c_2\alpha_2 + \cdots + c_n\alpha_n = 0
  \end{equation*}
  for $c_1,\dots, c_n$ in $F$ not all $0$. If we let
  \begin{equation*}
    X =
    \begin{bmatrix}
      c_1 \\ c_2 \\ \vdots \\ c_n
    \end{bmatrix},
  \end{equation*}
  then $AX = 0$ as required.
\end{proof}

\Exercise2 Let
\begin{equation*}
  \alpha_1 = (1, 1, -2, 1), \quad
  \alpha_2 = (3, 0, 4, -1), \quad
  \alpha_3 = (-1, 2, 5, 2).
\end{equation*}
Let
\begin{equation*}
  \alpha = (4, -5, 9, -7), \quad
  \beta = (3, 1, -4, 4), \quad
  \gamma = (-1, 1, 0, 1).
\end{equation*}
\begin{enumerate}
\item Which of the vectors $\alpha$, $\beta$, $\gamma$ are in the
  subspace of $R^4$ spanned by the $\alpha_i$?
  \begin{solution}
    Let $A$ be the $3\times4$ matrix whose $i$th row is $\alpha_i$. By
    performing row reduction on $A$, we get
    \begin{equation*}
      \begin{bmatrix}
        1 & 1 & -2 & 1 \\
        3 & 0 & 4 & -1 \\
        -1 & 2 & 5 & 2
      \end{bmatrix}
      \rightarrow
      \begin{bmatrix}
        1 & 0 & 0 & -3/13 \\
        0 & 1 & 0 & 14/13 \\
        0 & 0 & 1 & -1/13
      \end{bmatrix}.
    \end{equation*}
    A vector $\rho$ is in the row space of $A$ if and only if
    \begin{equation*}
      \rho = c_1\alpha_1 + c_2\alpha_2 + c_3\alpha_3
      = \left(c_1, c_2, c_3, -\frac{3}{13}c_1
        + \frac{14}{13}c_2 - \frac1{13}c_3\right).
    \end{equation*}
    Checking each of $\alpha$, $\beta$, and $\gamma$, we see that only
    $\alpha$ is in the row space. So $\alpha$ is in the subspace
    spanned by the $\alpha_i$, while $\beta$ and $\gamma$ are not in
    the subspace.
  \end{solution}
\item Which of the vectors $\alpha$, $\beta$, $\gamma$ are in the
  subspace of $C^4$ spanned by the $\alpha_i$?
  \begin{solution}
    Our work above is still valid in $C^4$. $\alpha_1$, $\alpha_2$,
    and $\alpha_3$ will span a larger subspace due to the scalars
    being taken from $C$ instead of $R$, but the members of this
    subspace will still have the same form as before. Thus, of the
    three vectors $\alpha$, $\beta$, and $\gamma$, only $\alpha$ is in
    the subspace.
  \end{solution}
\item Does this suggest a theorem?
  \begin{solution}
    This suggests the following theorem: let $F$ be a subfield of the
    field $E$. Let $\alpha$ be a vector in $F^n$, and let
    $\beta_1,\dots,\beta_n$ in $F^n$ span some subspace. Then $\alpha$
    is in this subspace of $F^n$ if and only if it is in the subspace
    of $E^n$ spanned by the same vectors $\beta_i$.
  \end{solution}
\end{enumerate}

\Exercise3 Consider the vectors in $R^4$ defined by
\begin{equation*}
  \alpha_1 = (-1, 0, 1, 2), \quad
  \alpha_2 = (3, 4, -2, 5), \quad
  \alpha_3 = (1, 4, 0, 9).
\end{equation*}
Find a system of homogeneous linear equations for which the space of
solutions is exactly the subspace of $R^4$ spanned by the three given
vectors.
\begin{solution}
  Let $A$ be the $3\times4$ matrix whose $i$th row is $\alpha_i$. We
  can perform row-reduction on $A$ to get
  \begin{equation*}
    R =
    \begin{bmatrix}
      1 & 0 & -1 & -2 \\[3pt]
      0 & 1 & \frac14 & \frac{11}4 \\[3pt]
      0 & 0 & 0 & 0
    \end{bmatrix}.
  \end{equation*}
  Then a vector $\rho$ in $R^4$ is in the row space of $A$ if and only
  if it has the form
  \begin{equation*}
    \rho = \left(r_1, r_2, \frac14r_2 - r_1, \frac{11}4r_2 - 2r_1\right),
  \end{equation*}
  where $r_1$ and $r_2$ are real numbers. If we label the components
  of $\rho$ as $(x_1,x_2,x_3,x_4)$, we get the following system of
  equations:
  \begin{align*}
    x_3 &= \frac14x_2 - x_1 \\
    x_4 &= \frac{11}4x_2 - 2x_1.
  \end{align*}
  Or, we can rearrange these equations to write
  \begin{alignat*}{5}
    x_1 &{}-{}& \frac14x_2 &{}+{}& x_3 && &{}={}& 0 \\
    2x_1 &{}-{}& \frac{11}4x_2 && &{}+{}& x_4 &{}={}& 0.
  \end{alignat*}
  This system is homogeneous and its solution set is precisely the
  subspace spanned by $\alpha_1$, $\alpha_2$, and $\alpha_3$.
\end{solution}
