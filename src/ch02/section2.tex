\section{Subspaces}

\Exercise1 Which of the following sets of vectors
$\alpha = (a_1,\dots,a_n)$ in $R^n$ are subspaces of $R^n$ ($n\geq3$)?
\begin{enumerate}
\item all $\alpha$ such that $a_1\geq0$
  \begin{solution}
    This is not a subspace since it is not closed under scalar
    multiplication (take any negative scalar).
  \end{solution}
\item all $\alpha$ such that $a_1 + 3a_2 = a_3$
  \begin{solution}
    This is a subspace: Let $\beta = (b_1,b_2,\dots,b_n)$. Then
    consider the vector $c\alpha + \beta$. We have
    \begin{align*}
      (ca_1 + b_1) + 3(ca_2 + b_2)
      &= c(a_1 + 3a_2) + (b_1 + 3b_2) \\
      &= ca_3 + b_3.
    \end{align*}
    Therefore $c\alpha + \beta$ is in the subset, so it is a subspace
    by Theorem~1.
  \end{solution}
\item all $\alpha$ such that $a_2 = a_1^2$
  \begin{solution}
    This is not a subspace since it is not closed under vector
    addition. For example, $(1, 1, 1, \dots)$ is in the set, but the
    sum of this vector with itself is not.
  \end{solution}
\item all $\alpha$ such that $a_1a_2 = 0$
  \begin{solution}
    This is not a subspace since it is not closed under vector
    addition. For example $(1,0,\dots)$ and $(0,1,\dots)$ are each in
    the set, but their sum is not.
  \end{solution}
\item all $\alpha$ such that $a_2$ is rational
  \begin{solution}
    This is not a subspace because it is not closed under scalar
    multiplication: Multiplication of any vector in the set having
    $a_2\neq0$ by an irrational scalar produces a vector that is not
    in the set.
  \end{solution}
\end{enumerate}

\Exercise2 Let $V$ be the (real) vector space of all functions $f$
from $R$ into $R$. Which of the following sets of functions are
subspaces of $V$?
\begin{enumerate}
\item all $f$ such that $f(x^2) = f(x)^2$
  \begin{solution}
    The functions $f$ and $g$ given by
    \begin{equation*}
      f(x) = x \quad\text{and}\quad g(x) = 1
    \end{equation*}
    each belong to this set, but their sum $f + g$ does not. Therefore
    this is not a subspace.
  \end{solution}
\item all $f$ such that $f(0) = f(1)$
  \begin{solution}
    Suppose $f$ and $g$ both belong to this set. Then
    \begin{align*}
      (cf + g)(0) &= cf(0) + g(0) \\
                  &= cf(1) + g(1) \\
                  &= (cf + g)(1),
    \end{align*}
    so the set satisfies the subspace criterion of Theorem~1 and is
    thus a subspace of $V$.
  \end{solution}
\item all $f$ such that $f(3) = 1 + f(-5)$
  \begin{solution}
    Take any $f$ and $g$ in this set. Then
    \begin{equation*}
      (f + g)(3) = f(3) + g(3) = 2 + (f + g)(-5),
    \end{equation*}
    which does not belong to the set. Therefore this set is not a
    subspace.
  \end{solution}
\item all $f$ such that $f(-1) = 0$
  \begin{solution}
    Let $f$ and $g$ be such functions. Then
    \begin{equation*}
      (cf + g)(-1) = cf(-1) + g(-1) = 0 + 0 = 0,
    \end{equation*}
    so this set is a subspace of $V$ by Theorem~1.
  \end{solution}
\item all $f$ which are continuous
  \begin{solution}
    If $f$ and $g$ are continuous, then $cf + g$ is also continuous,
    so this is a subspace.
  \end{solution}
\end{enumerate}

\Exercise3 Is the vector $(3,-1,0,-1)$ in the subspace of $R^5$
spanned by the vectors
\begin{equation*}
  (2,-1,3,2), \quad
  (-1,1,1,-3),
  \quad\text{and}\quad
  (1,1,9,-5)?
\end{equation*}
\begin{solution}
  The subspace spanned by these three vectors consists of all linear
  combinations
  \begin{equation*}
    x_1(2,-1,3,2) + x_2(-1,1,1,-3) + x_3(1,1,9,-5).
  \end{equation*}
  Therefore $(3,-1,0,-1)$ is in this subspace if and only if the
  system of equations
  \begin{alignat*}{3}
    2x_1 &{}-{}& x_2 &{}+{}& x_3 &{}= 3 \\
    -x_1 &{}+{}& x_2 &{}+{}& x_3 &{}= -1 \\
    3x_1 &{}+{}& x_2 &{}+{}& 9x_3 &{}= 0 \\
    2x_1 &{}-{}& 3x_2 &{}-{}& 5x_3 &{}= -1
  \end{alignat*}
  has a solution. However, the augmented matrix can be row-reduced to
  \begin{equation*}
    \begin{bmatrix}
      2 & -1 & 1 & 3 \\
      -1 & 1 & 1 & -1 \\
      3 & 1 & 9 & 0 \\
      2 & -3 & -5 & -1
    \end{bmatrix}
    \rightarrow
    \begin{bmatrix}
      1 & 0 & 2 & 0 \\
      0 & 1 & 3 & 0 \\
      0 & 0 & 0 & 1 \\
      0 & 0 & 0 & 0
    \end{bmatrix}.
  \end{equation*}
  Therefore, this system of equations has no solution and the vector
  $(3,-1,0,-1)$ is not in the subspace spanned by the other three
  given vectors.
\end{solution}

\Exercise4 Let $W$ be the set of all $(x_1,x_2,x_3,x_4,x_5)$ in $R^5$
which satisfy
\begin{alignat*}{6}
  2x_1 &{}-{}& x_2 &{}+{}& \tfrac43x_3 &{}-{}& x_4 && &{}={}& 0 \\
  x_1 && &{}+{}& \tfrac23x_3 && &{}-{}& x_5 &{}={}& 0 \\
  9x_1 &{}-{}& 3x_2 &{}+{}& 6x_3 &{}-{}& 3x_4 &{}-{}& 3x_5 &{}={}& 0\rlap.
\end{alignat*}
Find a finite set of vectors which spans $W$.
\begin{solution}
  After performing the necessary elementary row operations, the
  coefficient matrix becomes
  \begin{equation*}
    \begin{bmatrix}
      2 & -1 & \frac43 & -1 & 0 \\[3pt]
      1 & 0 & \frac23 & 0 & -1 \\[3pt]
      9 & -3 & 6 & -3 & -3
    \end{bmatrix}
    \rightarrow
    \begin{bmatrix}
      1 & 0 & \frac23 & 0 & -1 \\[3pt]
      0 & 1 & 0 & 1 & -2 \\[3pt]
      0 & 0 & 0 & 0 & 0
    \end{bmatrix}.
  \end{equation*}
  So, letting $x_3 = 3t$, $x_4 = u$, and $x_5 = v$, we see that the
  elements of $W$ have the form
  \begin{equation*}
    (v-2t, 2v-u, 3t, u, v).
  \end{equation*}
  Therefore, a spanning set for $W$ is given by the vectors
  \begin{equation*}
    (-2,0,3,0,0), \quad (0,-1,0,1,0), \quad
    \text{and}\quad
    (1,2,0,0,1). \qedhere
  \end{equation*}
\end{solution}

\Exercise5 Let $F$ be a field and let $n$ be a positive integer
($n\geq2$). Let $V$ be the vector space of all $n\times n$ matrices
over $F$. Which of the following sets of matrices $A$ in $V$ are
subspaces of $V$?
\begin{enumerate}
\item all invertible $A$
  \begin{solution}
    This cannot be a subspace since the zero matrix is not invertible.
  \end{solution}
\item all non-invertible $A$
  \begin{solution}
    This is also not a subspace since it is possible for the sum of
    two non-invertible matrices to be invertible. For example, in the
    $2\times2$ case, the matrices
    \begin{equation*}
      \begin{bmatrix}
        1 & 0 \\
        0 & 0
      \end{bmatrix}
      \quad\text{and}\quad
      \begin{bmatrix}
        0 & 0 \\
        0 & 1
      \end{bmatrix}
    \end{equation*}
    are not invertible, but their sum is the identity matrix, which is
    invertible.
  \end{solution}
\item all $A$ such that $AB = BA$, where $B$ is some fixed matrix in
  $V$
  \begin{solution}
    Let $A_1$ and $A_2$ be matrices in $V$ such that $A_1B = BA_1$ and
    $A_2B = BA_2$. Then, since matrix multiplication is distributive,
    \begin{align*}
      (cA_1 + A_2)B &= cA_1B + A_2B \\
                    &= cBA_1 + BA_2 \\
                    &= B(cA_1) + BA_2 \\
                    &= B(cA_1 + A_2).
    \end{align*}
    Therefore, by Theorem~1, this set is a subspace of $V$.
  \end{solution}
\item all $A$ such that $A^2 = A$
  \begin{solution}
    We will assume that the field $F$ has more than two elements. In
    that case, this set cannot be a subspace since the identity matrix
    has the property that $I^2 = I$, but the sum of the identity with
    itself does not have this property.
  \end{solution}
\end{enumerate}

\Exercise6
\begin{enumerate}
\item Prove that the only subspaces of $R^1$ are $R^1$ and the zero
  subspace.
  \begin{proof}
    Suppose $W$ is a subspace of $R^1$. If $W = \{0\}$ we are done, so
    suppose $W$ contains a nonzero element $x$. Then $W$ must contain
    $cx$ for any real number $c$. In particular, if $r$ is any real
    number, then $W$ must contain $r$ since $r = (rx^{-1})x$. This
    shows that $W = R^1$.
  \end{proof}
\item Prove that a subspace of $R^2$ is $R^2$, or the zero subspace,
  or consists of all scalar multiples of some fixed vector in
  $R^2$. (The last type of subspace is, intuitively, a straight line
  through the origin.)
  \begin{proof}
    Let $W$ be a subspace of $R^2$. If $W = \{(0,0)\}$ we are done, so
    assume $W$ contains a nonzero vector $\alpha$. Then $W$ must
    contain all scalar multiples of $\alpha$. If these are the only
    elements in $W$, then we are again finished. If, however, $W$
    contains two nonzero elements $\alpha$ and $\beta$ such that
    $\beta$ is {\em not} a scalar multiple of $\alpha$, then we must
    show that $W = R^2$.

    Let $\alpha = (a_1,a_2)$ and $\beta = (b_1,b_2)$. Also let
    $\gamma = (c_1,c_2)$ be any element in $R^2$. Then $\gamma$ is a
    linear combination of $\alpha$ and $\beta$ if and only if the
    system of equations
    \begin{alignat*}{3}
      a_1x_1 &{}+{}& b_1x_2 &{}={}& c_1 \\
      a_2x_1 &{}+{}& b_2x_2 &{}={}& c_2
    \end{alignat*}
    has a solution. Suppose the coefficient matrix
    \begin{equation*}
      \begin{bmatrix}
        a_1 & b_1 \\
        a_2 & b_2
      \end{bmatrix}
    \end{equation*}
    is not invertible. By
    Exercise~\ref{exercise:lin-eq:2-by-2-inv-crit}, we then know that
    $a_1b_2 - a_2b_1 = 0$. Now, one of $a_1$ and $a_2$ is nonzero. If
    $a_1\neq0$, then
    \begin{equation*}
      b_2 = \frac{b_1}{a_1}\cdot a_2.
    \end{equation*}
    Also
    \begin{equation*}
      b_1 = \frac{b_1}{a_1}\cdot a_1,
    \end{equation*}
    and we have a contradiction since $\beta$ was assumed to not be a
    scalar multiple of $\alpha$. Similarly $a_2\neq0$ also leads to a
    contradiction. This shows that the system of equations above has a
    solution, so that $W = R^2$.
  \end{proof}
\item Can you describe the subspaces of $R^3$?
  \begin{solution}
    The subspaces of $R^3$ are the zero subspace, the set of all
    scalar multiples of a fixed nonzero vector (i.e., a line through
    the origin), the set of all linear combinations of two linearly
    independent vectors (i.e., a plane through the origin), and $R^3$
    itself.
  \end{solution}
\end{enumerate}
