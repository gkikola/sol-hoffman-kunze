\section{Coordinates}

\Exercise1 Show that the vectors
\begin{align*}
  \alpha_1 &= (1, 1, 0, 0), \quad \alpha_2 = (0, 0, 1, 1) \\
  \alpha_3 &= (1, 0, 0, 4), \quad \alpha_4 = (0, 0, 0, 2)
\end{align*}
form a basis for $R^4$. Find the coordinates of each of the standard
basis vectors in the ordered basis
$\{\alpha_1, \alpha_2, \alpha_3, \alpha_4\}$.
\begin{solution}
  Let
  \begin{equation*}
    P =
    \begin{bmatrix}
      1 & 0 & 1 & 0 \\
      1 & 0 & 0 & 0 \\
      0 & 1 & 0 & 0 \\
      0 & 1 & 4 & 2
    \end{bmatrix}.
  \end{equation*}
  $P$ is invertible and has inverse
  \begin{equation*}
    P^{-1} =
    \begin{bmatrix}
      0 & 1 & 0 & 0 \\
      0 & 0 & 1 & 0 \\
      1 & -1 & 0 & 0 \\
      -2 & 2 & -\frac12 & \frac12
    \end{bmatrix}.
  \end{equation*}
  By Theorem~8, $\{\alpha_1, \alpha_2, \alpha_3, \alpha_4\}$ is a
  basis for $R^4$. Moreover, the $j$th column of $P^{-1}$ gives the
  coordinates of the standard basis vector $\epsilon_j$ in the ordered
  basis $\{\alpha_1, \alpha_2, \alpha_3, \alpha_4\}$.
\end{solution}

\Exercise2 Find the coordinate matrix of the vector $(1, 0, 1)$ in the
basis of $C^3$ consisting of the vectors $(2i, 1, 0)$, $(2, -1, 1)$,
$(0, 1 + i, 1 - i)$, in that order.
\begin{solution}
  Let
  \begin{equation*}
    P =
    \begin{bmatrix}
      2i & 2 & 0 \\
      1 & -1 & 1 + i \\
      0 & 1 & 1 - i
    \end{bmatrix}.
  \end{equation*}
  Then
  \begin{equation*}
    P^{-1} =
    \begin{bmatrix}
      \frac12 - \frac12i & -i & -1 \\[3pt]
      -\frac12i & -1 & i \\[3pt]
      -\frac14 + \frac14i & \frac12 + \frac12i & 1
    \end{bmatrix}.
  \end{equation*}
  Since
  \begin{equation*}
    P^{-1}
    \begin{bmatrix}
      1 \\ 0 \\ 1
    \end{bmatrix}
    =
    \begin{bmatrix}
      \frac12 - \frac12i & -i & -1 \\[3pt]
      -\frac12i & -1 & i \\[3pt]
      -\frac14 + \frac14i & \frac12 + \frac12i & 1
    \end{bmatrix}
    \begin{bmatrix}
      1 \\ 0 \\ 1
    \end{bmatrix}
    =
    \begin{bmatrix}
      -\frac12 - \frac12i \\[3pt] \frac12i \\[3pt] \frac34 + \frac14i
    \end{bmatrix},
  \end{equation*}
  the vector $(1,0,1)$ has coordinates
  $\left(-\frac12 - \frac12i, \frac12i, \frac34 + \frac14i\right)$ in
  the given basis.
\end{solution}

\Exercise3 Let $\mathcal{B} = \{\alpha_1, \alpha_2, \alpha_3\}$ be the
ordered basis for $R^3$ consisting of
\begin{equation*}
  \alpha_1 = (1, 0, -1), \quad
  \alpha_2 = (1, 1, 1) \quad
  \alpha_3 = (1, 0, 0).
\end{equation*}
What are the coordinates of the vector $(a, b, c)$ in the ordered
basis $\mathcal{B}$?
\begin{solution}
  Let
  \begin{equation*}
    P =
    \begin{bmatrix}
      1 & 1 & 1 \\
      0 & 1 & 0 \\
      -1 & 1 & 0
    \end{bmatrix}.
  \end{equation*}
  Then
  \begin{equation*}
    P^{-1} =
    \begin{bmatrix}
      0 & 1 & -1 \\
      0 & 1 & 0 \\
      1 & -2 & 1
    \end{bmatrix}
  \end{equation*}
  and
  \begin{equation*}
    P^{-1}
    \begin{bmatrix}
      a \\ b \\ c
    \end{bmatrix}
    =
    \begin{bmatrix}
      0 & 1 & -1 \\
      0 & 1 & 0 \\
      1 & -2 & 1
    \end{bmatrix}
    \begin{bmatrix}
      a \\ b \\ c
    \end{bmatrix}
    =
    \begin{bmatrix}
      b - c \\ b \\ a - 2b + c
    \end{bmatrix}.
  \end{equation*}
  So $(a,b,c)$ has coordinates $(b - c, b, a - 2b + c)$ in the ordered
  basis $\mathcal{B}$.
\end{solution}

\Exercise4 Let $W$ be the subspace of $C^3$ spanned by
$\alpha_1 = (1,0,i)$ and $\alpha_2 = (1 + i, 1, -1)$.
\begin{enumerate}
\item Show that $\alpha_1$ and $\alpha_2$ form a basis for $W$.
  \begin{solution}
    Since neither $\alpha_1$ nor $\alpha_2$ is a scalar multiple of
    the other, the set $\{\alpha_1, \alpha_2\}$ is linearly
    independent. Hence this set is a basis for $W$.
  \end{solution}

\item Show that the vectors $\beta_1 = (1,1,0)$ and
  $\beta_2 = (1,i,1+i)$ are in $W$ and form another basis for $W$.
  \begin{solution}
    If $c_1(1,0,i) + c_2(1 + i, 1, -1) = (1, 1, 0)$, then equating
    coordinates and solving the resulting system gives $c_1 = -i$ and
    $c_2 = 1$. Therefore $\beta_1$ is in $W$ and its coordinates in
    $\{\alpha_1,\alpha_2\}$ are $(-i,1)$.

    In a similar way, we can determine that $\beta_2$ is in $W$ and
    has coordinates $(2-i,i)$ in the given basis.

    Neither $\beta_1$ nor $\beta_2$ is a scalar multiple of the other,
    so the set $\{\beta_1,\beta_2\}$ is linearly independent. Since
    $W$ has dimension $2$, the set $\{\beta_1,\beta_2\}$ is also a
    basis for $W$.
  \end{solution}

\item What are the coordinates of $\alpha_1$ and $\alpha_2$ in the
  ordered basis $\{\beta_1, \beta_2\}$ for $W$?
  \begin{solution}
    From the coordinates for $\beta_1$ and $\beta_2$ that we found
    previously, we get the transition matrix
    \begin{equation*}
      P =
      \begin{bmatrix}
        -i & 2 - i \\
        1 & i
      \end{bmatrix}.
    \end{equation*}
    This matrix has inverse
    \begin{equation*}
      P^{-1} =
      \begin{bmatrix}
        \frac12 - \frac12i & \frac32 + \frac12i \\[3pt]
        \frac12 + \frac12i & -\frac12 + \frac12i
      \end{bmatrix},
    \end{equation*}
    so
    \begin{align*}
      \alpha_1 &= \left(\frac12 - \frac12i\right)\beta_1
      + \left(\frac12 + \frac12i\right)\beta_2 \\
      \intertext{and}
      \alpha_2 &= \left(\frac32 + \frac12i\right)\beta_1
      + \left(-\frac12 + \frac12i\right)\beta_2. \qedhere
    \end{align*}
  \end{solution}
\end{enumerate}

% \Exercise5 Let $\alpha = (x_1, x_2)$ and $\beta = (y_1, y_2)$ be
% vectors in $R^2$ such that
% \begin{equation*}
%   x_1y_1 + x_2y_2 = 0, \quad
%   x_1^2 + x_2^2 = y_1^2 + y_2^2 = 1.
% \end{equation*}
% Prove that $\mathcal{B} = \{\alpha,\beta\}$ is a basis for $R^2$. Find
% the coordinates of the vector $(a, b)$ in the ordered basis
% $\mathcal{B} = \{\alpha, \beta\}$.
% \begin{solution}
%   Let $c_1$ and $c_2$ be scalars in $R$ such that
%   \begin{equation*}
%     c_1\alpha + c_2\beta = 0.
%   \end{equation*}
%   This leads to the system of equations
%   \begin{align*}
%     c_1x_1 + c_2y_1 &= 0 \\
%     c_1x_2 + c_2y_2 &= 0.
%   \end{align*}
%   Multiplying the first equation by $x_1$ and the second equation by
%   $x_2$ gives
%   \begin{align*}
%     c_1x_1^2 + c_2x_1y_1 &= 0 \\
%     c_1x_2^2 + c_2x_2y_2 &= 0.
%   \end{align*}
%   Adding these two equations then gives
%   \begin{equation*}
%     c_1(x_1^2 + x_2^2) + c_2(x_1y_1 + x_2y_2) = 0.
%   \end{equation*}
%   Since $x_1y_1 + x_2y_2 = 0$ and $x_1^2 + x_2^2 = 1$, the above
%   equation simplifies to $c_1 = 0$.

%   By a symmetric argument, we can determine that $c_2 = 0$
%   also. Therefore $\mathcal{B}$ is linearly independent and, since
%   $\dim R^2 = 2$, $\mathcal{B}$ must span $R^2$. So $\mathcal{B}$ is a
%   basis for $R^2$.

%   To find the coordinates for $(a,b)$, we will perform row-reduction
%   on the augmented matrix
%   \begin{equation*}
%     \begin{bmatrix}
%       x_1 & y_1 & a \\
%       x_2 & y_2 & b
%     \end{bmatrix}.
%   \end{equation*}
% \end{solution}

\Exercise6 Let $V$ be the vector space over the complex numbers of all
functions from $R$ into $C$, i.e., the space of all complex-valued
functions on the real line. Let $f_1(x) = 1$, $f_2(x) = e^{ix}$,
$f_3(x) = e^{-ix}$.
\begin{enumerate}
\item Prove that $f_1$, $f_2$, and $f_3$ are linearly independent.
  \begin{proof}
    Let $c_1$, $c_2$, and $c_3$ be complex numbers such that
    \begin{equation*}
      c_1f_1(x) + c_2f_2(x) + c_3f_3(x) = 0
    \end{equation*}
    for all $x$ in $R$. Then
    \begin{equation*}
      c_1 + c_2e^{ix} + c_3e^{-ix} = 0.
    \end{equation*}
    Using Euler's formula, we can write
    \begin{equation*}
      c_1 + c_2(\cos x + i\sin x) + c_3(\cos x - i\sin x) = 0
    \end{equation*}
    or, rearranging,
    \begin{equation}
      \label{eq:vec-spaces:cplx-funcs-lin-comb}
      c_1 + (c_2 + c_3)\cos x + (c_2 - c_3)i\sin x = 0.
    \end{equation}
    If $x = 0$, then
    \begin{equation}
      \label{eq:vec-spaces:cplx-funcs-x-equals-0}
      c_1 + c_2 + c_3 = 0,
    \end{equation}
    while if $x = \pi$ we get
    \begin{equation}
      \label{eq:vec-spaces:cplx-funcs-x-equals-pi}
      c_1 - c_2 - c_3 = 0.
    \end{equation}
    Equations \eqref{eq:vec-spaces:cplx-funcs-x-equals-0} and
    \eqref{eq:vec-spaces:cplx-funcs-x-equals-pi} together imply that
    $c_1 = 0$.

    Next, letting $x = \pi/2$ in
    \eqref{eq:vec-spaces:cplx-funcs-lin-comb}, we get
    \begin{equation}
      \label{eq:vec-spaces:cplx-funcs-x-equals-pi-2}
      (c_2 - c_3)i = 0,
    \end{equation}
    which implies that $c_2 = c_3$. Equation
    \eqref{eq:vec-spaces:cplx-funcs-x-equals-0} then implies that
    $c_2 = c_3 = 0$.

    Since it is necessary that $c_1 = c_2 = c_3 = 0$, it follows that
    $\{f_1,f_2,f_3\}$ is a linearly independent set.
  \end{proof}
\item Let $g_1(x) = 1$, $g_2(x) = \cos x$, $g_3(x) = \sin x$. Find an
  invertible $3\times3$ matrix $P$ such that
  \begin{equation*}
    g_j = \sum_{i=1}^3 P_{ij}f_i.
  \end{equation*}
  \begin{solution}
    First, we have $g_1 = f_1$. Next, since
    \begin{equation*}
      f_2(x) + f_3(x) = (\cos x + i\sin x) + (\cos x - i\sin x)
      = 2\cos x,
    \end{equation*}
    we have $g_2 = \frac12f_2 + \frac12f_3$. And since
    \begin{equation*}
      f_2(x) - f_3(x) = 2i\sin x,
    \end{equation*}
    we see that $g_3 = -\frac{i}2f_2 + \frac{i}2f_3$. Therefore the
    desired matrix is
    \begin{equation*}
      P =
      \begin{bmatrix}
        1 & 0 & 0 \\[3pt]
        0 & \frac12 & -\frac12i \\[3pt]
        0 & \frac12 & \frac12i
      \end{bmatrix},
    \end{equation*}
    and this matrix is invertible.
  \end{solution}
\end{enumerate}
