\chapter{Linear Equations}

\skipsection
\section{Systems of Linear Equations}

\Exercise1 Verify that the set of complex numbers described in
Example~4 is a subfield of $C$.
\begin{solution}
  The set in Example~4 consisted of all complex numbers of the form
  $x + y\sqrt2$, where $x$ and $y$ are rational. Call this set $F$.

  Note that $0 = 0 + 0\sqrt2\in F$ and $1 = 1 + 0\sqrt2\in F$. If
  $\alpha = a + b\sqrt2$ and $\beta = c + d\sqrt2$ are any elements of
  $F$, then
  \begin{equation*}
    \alpha + \beta = (a + c) + (b + d)\sqrt2 \in F,
  \end{equation*}
  and
  \begin{equation*}
    -\alpha = -a - b\sqrt2 \in F.
  \end{equation*}
  We also have
  \begin{align*}
    \alpha\beta
    &= ac + ad\sqrt2 + bc\sqrt2 + 2bd \\
    &= (ac + 2bd) + (ad + bc)\sqrt2 \in F
  \end{align*}
  and, provided $\alpha$ is nonzero,
  \begin{equation*}
    \alpha^{-1}
    = \frac1{a + b\sqrt2}
    = \frac{a - b\sqrt2}{a^2 - 2b^2}
    = \frac{a}{a^2 - 2b^2} - \frac{b}{a^2 - 2b^2}\sqrt2 \in F.
  \end{equation*}
  Since $F$ contains $0$ and $1$ and is closed under addition,
  multiplication, additive inverses, and multiplicative inverses, $F$
  is a subfield of $C$.
\end{solution}

\Exercise2
\label{exercise:lin-eq:equiv-systems}
Let $F$ be the field of complex numbers. Are the following two systems
of linear equations equivalent? If so, express each equation in each
system as a linear combination of the equations in the other system.
\begin{alignat*}{4}
  x_1  &{}- x_2 &{}= 0 &\qquad\qquad& 3x_1 &{}+ x_2 &{}= 0 \\
  2x_1 &{}+ x_2 &{}= 0 &&        x_1 &{}+ x_2 &{}= 0
\end{alignat*}
\begin{solution}
  The systems are equivalent. For the first system, we can write
  \begin{align*}
    x_1 - x_2 &= (3x_1 + x_2) - 2(x_1 + x_2) = 0, \\
    2x_1 + x_2 &= \tfrac12(3x_1 + x_2) + \tfrac12(x_1 + x_2) = 0.
  \end{align*}
  And for the second system,
  \begin{align*}
    3x_1 + x_2 &= \tfrac13(x_1 - x_2) + \tfrac43(2x_1 + x_2) = 0, \\
    x_1 + x_2 &= -\tfrac13(x_1 - x_2) + \tfrac23(2x_1 + x_2) = 0. \qedhere
  \end{align*}
\end{solution}

\Exercise3 Test the following systems of equations as in
Exercise~\ref{exercise:lin-eq:equiv-systems}.
\begin{alignat*}{8}
  -x_1 &{}+{}& x_2 &{}+{}& 4x_3 &{}={}& 0
  &\qquad\qquad& x_1 &\phantom{{}+{}}& &{}-{}&& x_3 &{}={}& 0 \\
  x_1 &{}+{}& 3x_2 &{}+{}& 8x_3 &{}={}& 0
  && && x_2 &{}+{}&& 3x_3 &{}={}& 0 \\
  \tfrac12x_1 &{}+{}& x_2 &{}+{}& \tfrac52x_3 &{}={}& 0
\end{alignat*}
\begin{solution}
  For the first system, we have
  \begin{align*}
    -x_1 + x_2 + 4x_3 &= -(x_1 - x_3) + (x_2 + 3x_3) = 0, \\
    x_1 + 3x_2 + 8x_3 &= (x_1 - x_3) + 3(x_2 + 3x_3) = 0, \\
    \tfrac12x_1 + x_2 + \tfrac52x_3 &= \tfrac12(x_1 - x_3) + (x_2 + 3x_3) = 0.
  \end{align*}
  For the second system, we have
  \begin{align*}
    x_1 - x_3
    &= -\tfrac34(-x_1 + x_2 + 4x_3) + \tfrac14(x_1 + 3x_2 + 8x_3)
      + 0(\tfrac12x_1 + x_2 + \tfrac52x_3), \\
    x_2 + 3x_3
    &= \tfrac14(-x_1 + x_2 + 4x_3) + \tfrac14(x_1 + 3x_2 + 8x_3)
      + 0(\tfrac12x_1 + x_2 + \tfrac52x_3).
  \end{align*}
  So, the two systems are equivalent.
\end{solution}

\Exercise4 Test the following systems as in
Exercise~\ref{exercise:lin-eq:equiv-systems}.
\begin{alignat*}{10}
  2x_1 &{}+{}& (-1 + i)x_2 &     &       &{}+{}&  x_4 &{}={}& 0
  &\qquad& \left(1+\frac{i}2\right)x_1 &{}+{}&       8x_2
  &{}-{}& ix_3 &{}-{}&  x_4 &{}={}& 0 \\
       &     &        3x_2 &{}-{}& 2ix_3 &{}+{}& 5x_4 &{}={}& 0
  &      &                  \tfrac23x_1 &{}-{}& \tfrac12x_2
  &{}+{}&  x_3 &{}+{}& 7x_4 &{}={}& 0
\end{alignat*}
\begin{solution}
  Call the equations in the system on the left $L_1$ and $L_2$, and
  the equations on the right $R_1$ and $R_2$. If $R_1 = aL_1 + bL_2$
  then, by equating the coefficients of $x_3$, we get
  \begin{equation*}
    -i = -2ib,
  \end{equation*}
  which implies that $b = 1/2$. By equating the coefficients of $x_1$,
  we get
  \begin{equation*}
    1 + \frac{i}2 = 2a,
  \end{equation*}
  so that
  \begin{equation*}
    a = \frac12 + \frac14i.
  \end{equation*}
  Now, comparing the coefficients of $x_4$, we find that
  \begin{equation*}
    -1 = a + 5b = \frac12 + \frac14i + \frac52
    = 3 + \frac14i,
  \end{equation*}
  which is clearly a contradiction. Therefore the two systems are {\em
    not} equivalent.
\end{solution}

\Exercise5
\label{exercise:lin-eq:F2}
Let $F$ be a set which contains exactly two elements, $0$ and
$1$. Define an addition and multiplication by the tables:
\begin{center}
  \begin{tabular}{c|rr}
    $+$ & 0 & 1 \\
    \hline
    0 & 0 & 1 \\
    1 & 1 & 0
  \end{tabular}
  \qquad\qquad
  \begin{tabular}{c|rr}
    $\cdot$ & 0 & 1 \\
    \hline
    0 & 0 & 0 \\
    1 & 0 & 1
  \end{tabular}
\end{center}
Verify that the set $F$, together with these two operations, is a
field.
\begin{solution}
  From the symmetry in the tables, we see that both operations are
  commutative.

  By considering all eight possibilities, one can see that
  $(a + b) + c = a + (b + c)$. And one may in a similar way verify
  that $(ab)c = a(bc)$, so that associativity holds for the two
  operations.

  $0 + 0 = 0$ and $0 + 1 = 1$ so $F$ has an additive
  identity. Similarly, $1\cdot0 = 0$ and $1\cdot1 = 1$ so $F$ has a
  multiplicative identity.

  The additive inverse of $0$ is $0$ and the additive inverse of $1$
  is $1$. The multiplicative inverse of $1$ is $1$. So $F$ has
  inverses.

  Lastly, by considering the eight cases, one may verify that
  $a(b + c) = ab + ac$. Therefore distributivity of multiplication
  over addition holds and $F$ is a field.
\end{solution}

\Exercise7 Prove that each subfield of the field of complex numbers
contains every rational number.
\begin{proof}
  Let $F$ be a subfield of $C$ and let $r = m/n$ be any rational
  number, written in lowest terms. $F$ must contain $0$ and $1$, so if
  $r = 0$ then we are done. Now assume $r$ is nonzero.

  Since $1\in F$, and $F$ is closed under addition, we know that
  $1 + 1 = 2\in F$. And, if the integer $k$ is in $F$, then $k+1$ is
  also in $F$. By induction, we see that all positive integers belong
  to $F$. We also know that all negative integers are in $F$ because
  $F$ is closed under additive inverses. So, in particular, $m\in F$
  and $n\in F$.

  Now $F$ is closed under multiplicative inverses, so $n\in F$ implies
  $1/n\in F$. Finally, closure under multiplication shows that
  $m\cdot(1/n) = m/n = r\in F$. Since $r$ was arbitrary, we can
  conclude that all rational numbers are in $F$.
\end{proof}

\Exercise8 Prove that each field of characteristic zero contains a
copy of the rational number field.
\begin{proof}
  Let $F$ be a field of characteristic zero. Define the map
  $f\colon Q\to F$ (where $Q$ denotes the rational numbers) as
  follows. Let $f(0) = 0_F$ and $f(1) = 1_F$, where $0_F$ and $1_F$
  are the additive and multiplicative identities, respectively, of
  $F$. Given a positive integer $n$, define $f(n) = f(n - 1) + 1_F$
  and $f(-n) = -f(n)$. If a rational number $r = m/n$ is not an
  integer, define $f(r) = f(m)\cdot\left(f(n)\right)^{-1}$.

  First we show that the function $f$ preserves addition and
  multiplication. A simple induction argument will show that, in the
  case of integers $m$ and $n$, we have
  \begin{equation*}
    f(m + n) = f(m) + f(n)
    \quad\text{and}\quad
    f(mn) = f(m)f(n).
  \end{equation*}
  Now let $r_1 = m_1/n_1$ and $r_2 = m_2/n_2$ be rational numbers in
  lowest terms. Then, by the definition of $f$,
  \begin{align*}
    f(r_1 + r_2)
    &= f((m_1n_2 + m_2n_1)/(n_1n_2)) \\
    &= f(m_1n_2 + m_2n_1)f(n_1n_2)^{-1} \\
    &= (f(m_1)f(n_2) + f(m_2)f(n_1))f(n_1)^{-1}f(n_2)^{-1} \\
    &= f(m_1)f(n_1)^{-1} + f(m_2)f(n_2)^{-1} \\
    &= f(r_1) + f(r_2).
  \end{align*}
  Likewise,
  \begin{equation*}
    f(r_1r_2) = f(m_1)f(m_2)f(n_1)^{-1}f(n_2)^{-1}
    = f(r_1)f(r_2).
  \end{equation*}
  (Formally, this shows that $f$ is a ring homomorphism.)

  We will next show that the function $f$ is one-to-one. If
  $r_1 = m_1/n_1$ and $r_2 = m_2/n_2$ are rational numbers in lowest
  terms, then $f(r_1) = f(r_2)$ implies
  \begin{equation*}
    f(m_1)f(n_1)^{-1} = f(m_2)f(n_2)^{-1}
  \end{equation*}
  or
  \begin{equation*}
    f(m_1)f(n_2) = f(m_2)f(n_1).
  \end{equation*}
  This implies
  \begin{equation*}
    f(m_1n_2) = f(m_2n_1).
  \end{equation*}
  Now if $m_1n_2\neq m_2n_1$, then this would imply that $F$ does not
  have characteristic zero. So $m_1n_2 = m_2n_1$ and so $r_1 = r_2$.

  What we have shown is that every rational number corresponds to a
  distinct element of $F$, and that the operations of addition and
  multiplication of rational numbers is preserved by this
  correspondence. So $F$ contains a copy of $Q$.
\end{proof}
