\chapter{Linear Equations}

\skipsection
\section{Systems of Linear Equations}

\Exercise1 Verify that the set of complex numbers described in
Example~4 is a subfield of $C$.
\begin{solution}
  The set in Example~4 consisted of all complex numbers of the form
  $x + y\sqrt2$, where $x$ and $y$ are rational. Call this set $F$.

  Note that $0 = 0 + 0\sqrt2\in F$ and $1 = 1 + 0\sqrt2\in F$. If
  $\alpha = a + b\sqrt2$ and $\beta = c + d\sqrt2$ are any elements of
  $F$, then
  \begin{equation*}
    \alpha + \beta = (a + c) + (b + d)\sqrt2 \in F,
  \end{equation*}
  and
  \begin{equation*}
    -\alpha = -a - b\sqrt2 \in F.
  \end{equation*}
  We also have
  \begin{align*}
    \alpha\beta
    &= ac + ad\sqrt2 + bc\sqrt2 + 2bd \\
    &= (ac + 2bd) + (ad + bc)\sqrt2 \in F
  \end{align*}
  and, provided $\alpha$ is nonzero,
  \begin{equation*}
    \alpha^{-1}
    = \frac1{a + b\sqrt2}
    = \frac{a - b\sqrt2}{a^2 - 2b^2}
    = \frac{a}{a^2 - 2b^2} - \frac{b}{a^2 - 2b^2}\sqrt2 \in F.
  \end{equation*}
  Since $F$ contains $0$ and $1$ and is closed under addition,
  multiplication, additive inverses, and multiplicative inverses, $F$
  is a subfield of $C$.
\end{solution}

\Exercise2
\label{exercise:lin-eq:equiv-systems}
Let $F$ be the field of complex numbers. Are the following two systems
of linear equations equivalent? If so, express each equation in each
system as a linear combination of the equations in the other system.
\begin{alignat*}{4}
  x_1  &{}- x_2 &{}= 0 &\qquad\qquad& 3x_1 &{}+ x_2 &{}= 0 \\
  2x_1 &{}+ x_2 &{}= 0 &&        x_1 &{}+ x_2 &{}= 0
\end{alignat*}
\begin{solution}
  The systems are equivalent. For the first system, we can write
  \begin{align*}
    x_1 - x_2 &= (3x_1 + x_2) - 2(x_1 + x_2) = 0, \\
    2x_1 + x_2 &= \tfrac12(3x_1 + x_2) + \tfrac12(x_1 + x_2) = 0.
  \end{align*}
  And for the second system,
  \begin{align*}
    3x_1 + x_2 &= \tfrac13(x_1 - x_2) + \tfrac43(2x_1 + x_2) = 0, \\
    x_1 + x_2 &= -\tfrac13(x_1 - x_2) + \tfrac23(2x_1 + x_2) = 0. \qedhere
  \end{align*}
\end{solution}

\Exercise3 Test the following systems of equations as in
Exercise~\ref{exercise:lin-eq:equiv-systems}.
\begin{alignat*}{8}
  -x_1 &{}+{}& x_2 &{}+{}& 4x_3 &{}={}& 0
  &\qquad\qquad& x_1 &\phantom{{}+{}}& &{}-{}&& x_3 &{}={}& 0 \\
  x_1 &{}+{}& 3x_2 &{}+{}& 8x_3 &{}={}& 0
  && && x_2 &{}+{}&& 3x_3 &{}={}& 0 \\
  \tfrac12x_1 &{}+{}& x_2 &{}+{}& \tfrac52x_3 &{}={}& 0
\end{alignat*}
\begin{solution}
  For the first system, we have
  \begin{align*}
    -x_1 + x_2 + 4x_3 &= -(x_1 - x_3) + (x_2 + 3x_3) = 0, \\
    x_1 + 3x_2 + 8x_3 &= (x_1 - x_3) + 3(x_2 + 3x_3) = 0, \\
    \tfrac12x_1 + x_2 + \tfrac52x_3 &= \tfrac12(x_1 - x_3) + (x_2 + 3x_3) = 0.
  \end{align*}
  For the second system, we have
  \begin{align*}
    x_1 - x_3
    &= -\tfrac34(-x_1 + x_2 + 4x_3) + \tfrac14(x_1 + 3x_2 + 8x_3)
      + 0(\tfrac12x_1 + x_2 + \tfrac52x_3), \\
    x_2 + 3x_3
    &= \tfrac14(-x_1 + x_2 + 4x_3) + \tfrac14(x_1 + 3x_2 + 8x_3)
      + 0(\tfrac12x_1 + x_2 + \tfrac52x_3).
  \end{align*}
  So, the two systems are equivalent.
\end{solution}

\Exercise4 Test the following systems as in
Exercise~\ref{exercise:lin-eq:equiv-systems}.
\begin{alignat*}{10}
  2x_1 &{}+{}& (-1 + i)x_2 &     &       &{}+{}&  x_4 &{}={}& 0
  &\qquad& \left(1+\frac{i}2\right)x_1 &{}+{}&       8x_2
  &{}-{}& ix_3 &{}-{}&  x_4 &{}={}& 0 \\
       &     &        3x_2 &{}-{}& 2ix_3 &{}+{}& 5x_4 &{}={}& 0
  &      &                  \tfrac23x_1 &{}-{}& \tfrac12x_2
  &{}+{}&  x_3 &{}+{}& 7x_4 &{}={}& 0
\end{alignat*}
\begin{solution}
  Call the equations in the system on the left $L_1$ and $L_2$, and
  the equations on the right $R_1$ and $R_2$. If $R_1 = aL_1 + bL_2$
  then, by equating the coefficients of $x_3$, we get
  \begin{equation*}
    -i = -2ib,
  \end{equation*}
  which implies that $b = 1/2$. By equating the coefficients of $x_1$,
  we get
  \begin{equation*}
    1 + \frac{i}2 = 2a,
  \end{equation*}
  so that
  \begin{equation*}
    a = \frac12 + \frac14i.
  \end{equation*}
  Now, comparing the coefficients of $x_4$, we find that
  \begin{equation*}
    -1 = a + 5b = \frac12 + \frac14i + \frac52
    = 3 + \frac14i,
  \end{equation*}
  which is clearly a contradiction. Therefore the two systems are {\em
    not} equivalent.
\end{solution}
