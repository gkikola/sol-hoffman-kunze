\chapter{Linear Equations}

\skipsection
\section{Systems of Linear Equations}

\Exercise1 Verify that the set of complex numbers described in
Example~4 is a subfield of $C$.
\begin{solution}
  The set in Example~4 consisted of all complex numbers of the form
  $x + y\sqrt2$, where $x$ and $y$ are rational. Call this set $F$.

  Note that $0 = 0 + 0\sqrt2\in F$ and $1 = 1 + 0\sqrt2\in F$. If
  $\alpha = a + b\sqrt2$ and $\beta = c + d\sqrt2$ are any elements of
  $F$, then
  \begin{equation*}
    \alpha + \beta = (a + c) + (b + d)\sqrt2 \in F,
  \end{equation*}
  and
  \begin{equation*}
    -\alpha = -a - b\sqrt2 \in F.
  \end{equation*}
  We also have
  \begin{align*}
    \alpha\beta
    &= ac + ad\sqrt2 + bc\sqrt2 + 2bd \\
    &= (ac + 2bd) + (ad + bc)\sqrt2 \in F
  \end{align*}
  and, provided $\alpha$ is nonzero,
  \begin{equation*}
    \alpha^{-1}
    = \frac1{a + b\sqrt2}
    = \frac{a - b\sqrt2}{a^2 - 2b^2}
    = \frac{a}{a^2 - 2b^2} - \frac{b}{a^2 - 2b^2}\sqrt2 \in F.
  \end{equation*}
  Since $F$ contains $0$ and $1$ and is closed under addition,
  multiplication, additive inverses, and multiplicative inverses, $F$
  is a subfield of $C$.
\end{solution}

\Exercise2 Let $F$ be the field of complex numbers. Are the following
two systems of linear equations equivalent? If so, express each
equation in each system as a linear combination of the equations in
the other system.
\begin{alignat*}{4}
  x_1  &{}- x_2 &{}= 0 &\qquad\qquad& 3x_1 &{}+ x_2 &{}= 0 \\
  2x_1 &{}+ x_2 &{}= 0 &&        x_1 &{}+ x_2 &{}= 0
\end{alignat*}
\begin{solution}
  The systems are equivalent. For the first system, we can write
  \begin{align*}
    x_1 - x_2 &= (3x_1 + x_2) - 2(x_1 + x_2) = 0, \\
    2x_1 + x_2 &= \frac12(3x_1 + x_2) + \frac12(x_1 + x_2) = 0.
  \end{align*}
  And for the second system,
  \begin{align*}
    3x_1 + x_2 &= \frac13(x_1 - x_2) + \frac43(2x_1 + x_2) = 0, \\
    x_1 + x_2 &= -\frac13(x_1 - x_2) + \frac23(2x_1 + x_2) = 0. \qedhere
  \end{align*}
\end{solution}
