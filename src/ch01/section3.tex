\section{Matrices and Elementary Row Operations}

\Exercise1 Find all solutions to the system of equations
\begin{equation}
  \label{eq:lin-eq:elem-c-2-2}
  \begin{split}
    (1 - i)x_1 - ix_2 &= 0 \\
    2x_1 + (1 - i)x_2 &= 0.
  \end{split}
\end{equation}
\begin{solution}
  Using elementary row operations, we get
  \begin{equation*}
    \begin{bmatrix}
      1 - i & -i \\
      2 & 1 - i
    \end{bmatrix}
    \xrightarrow{(1)}
    \begin{bmatrix}
      1 - i & -i \\
      1 & \frac12 - \frac12i
    \end{bmatrix}
    \xrightarrow{(2)}
    \begin{bmatrix}
      0 & 0 \\
      1 & \frac12 - \frac12i
    \end{bmatrix}.
  \end{equation*}
  So the system in \eqref{eq:lin-eq:elem-c-2-2} is equivalent to
  \begin{equation*}
    x_1 + \left(\frac12 - \frac12i\right)x_2 = 0.
  \end{equation*}
  Therefore, if $c$ is any complex scalar, then $x_1 = (-1 + i)c$ and
  $x_2 = 2c$ is a solution to \eqref{eq:lin-eq:elem-c-2-2}.
\end{solution}

\Exercise2 If
\begin{equation*}
  A =
  \begin{bmatrix}
    3 & -1 & 2 \\
    2 & 1 & 1 \\
    1 & -3 & 0
  \end{bmatrix}
\end{equation*}
find all solutions of $AX = 0$ by row-reducing $A$.
\begin{solution}
  We get
  \begin{gather*}
    \begin{bmatrix}
      3 & -1 & 2 \\[3pt]
      2 & 1 & 1 \\[3pt]
      1 & -3 & 0
    \end{bmatrix}
    \xrightarrow{(1)}
    \begin{bmatrix}
      1 & -\frac13 & \frac23 \\[3pt]
      2 & 1 & 1 \\[3pt]
      1 & -3 & 0
    \end{bmatrix}
    \xrightarrow{(2)}
    \begin{bmatrix}
      1 & -\frac13 & \frac23 \\[3pt]
      0 & \frac53 & -\frac13 \\[3pt]
      0 & -\frac83 & -\frac23
    \end{bmatrix}
    \xrightarrow{(1)} \\
    \begin{bmatrix}
      1 & -\frac13 & \frac23 \\[3pt]
      0 & 1 & -\frac15 \\[3pt]
      0 & -\frac83 & -\frac23
    \end{bmatrix}
    \xrightarrow{(2)}
    \begin{bmatrix}
      1 & 0 & \frac35 \\[3pt]
      0 & 1 & -\frac15 \\[3pt]
      0 & 0 & -\frac65
    \end{bmatrix}
    \xrightarrow{(1)}
    \begin{bmatrix}
      1 & 0 & \frac35 \\[3pt]
      0 & 1 & -\frac15 \\[3pt]
      0 & 0 & 1
    \end{bmatrix}
    \xrightarrow{(2)} \\
    \begin{bmatrix}
      1 & 0 & 0 \\
      0 & 1 & 0 \\
      0 & 0 & 1
    \end{bmatrix}.
  \end{gather*}
  Thus $AX = 0$ has only the trivial solution.
\end{solution}

\Exercise3 If
\begin{equation*}
  A =
  \begin{bmatrix}
    6 & -4 & 0 \\
    4 & -2 & 0 \\
    -1 & 0 & 3
  \end{bmatrix}
\end{equation*}
find all solutions of $AX = 2X$ and all solutions of $AX = 3X$. (The
symbol $cX$ denotes the matrix each entry of which is $c$ times the
corresponding entry of $X$.)
\begin{solution}
  The matrix equation $AX = 2X$ corresponds to the system of linear
  equations
  \begin{alignat*}{4}
    6x_1 &{}-{}& 4x_2 && &{}={}& 2x_1 \\
    4x_1 &{}-{}& 2x_2 && &{}={}& 2x_2 \\
    -1x_1 && &{}+{}& 3x_3 &{}={}& 2x_3\rlap,
  \end{alignat*}
  or, equivalently,
  \begin{alignat*}{4}
    4x_1 &{}-{}& 4x_2 && &{}={}& 0 \\
    4x_1 &{}-{}& 4x_2 && &{}={}& 0 \\
    -1x_1 && &{}+{}& x_3 &{}={}& 0\rlap.
  \end{alignat*}
  This latter system is homogeneous, and can be represented by the
  equation $BX = 0$, where $B$ is given by
  \begin{equation*}
    B =
    \begin{bmatrix}
      4 & -4 & 0 \\
      4 & -4 & 0 \\
      -1 & 0 & 1
    \end{bmatrix}.
  \end{equation*}
  $B$ can be row-reduced:
  \begin{equation*}
    \begin{bmatrix}
      4 & -4 & 0 \\
      4 & -4 & 0 \\
      -1 & 0 & 1
    \end{bmatrix}
    \rightarrow
    \begin{bmatrix}
      1 & 0 & -1 \\
      0 & 1 & -1 \\
      0 & 0 & 0
    \end{bmatrix}.
  \end{equation*}
  Therefore any solution of $AX = 2X$ will have the form
  \begin{equation*}
    (x_1,x_2,x_3) = (a,a,a) = a(1,1,1),
  \end{equation*}
  where $a$ is a scalar.

  Similarly, the equation $AX = 3X$ can be solved by row-reducing
  \begin{equation*}
    \begin{bmatrix}
      3 & -4 & 0 \\
      4 & -5 & 0 \\
      -1 & 0 & 0
    \end{bmatrix}
    \rightarrow
    \begin{bmatrix}
      1 & 0 & 0 \\
      0 & 1 & 0 \\
      0 & 0 & 0
    \end{bmatrix}.
  \end{equation*}
  So, solutions of $AX = 3X$ have the form
  \begin{equation*}
    (x_1,x_2,x_3) = (0,0,b) = b(0,0,1),
  \end{equation*}
  where $b$ is a scalar.
\end{solution}

\Exercise4 Find a row-reduced matrix which is row-equivalent to
\begin{equation*}
  A =
  \begin{bmatrix}
    i & -(1 + i) & 0 \\
    1 & -2 & 1 \\
    1 & 2i & -1
  \end{bmatrix}.
\end{equation*}
\begin{solution}
  Using the elementary row operations, we get
  \begin{gather*}
    \begin{bmatrix}
      i & -(1 + i) & 0 \\
      1 & -2 & 1 \\
      1 & 2i & -1
    \end{bmatrix}
    \xrightarrow{(1)}
    \begin{bmatrix}
      1 & -1 + i & 0 \\
      1 & -2 & 1 \\
      1 & 2i & -1
    \end{bmatrix}
    \xrightarrow{(2)}
    \begin{bmatrix}
      1 & -1 + i & 0 \\
      0 & -1 - i & 1 \\
      0 & 1+i & -1
    \end{bmatrix}
    \xrightarrow{(1)} \\
    \begin{bmatrix}
      1 & -1 + i & 0 \\[3pt]
      0 & 1 & -\frac12 + \frac12i \\[3pt]
      0 & 1+i & -1
    \end{bmatrix}
    \xrightarrow{(2)}
    \begin{bmatrix}
      1 & 0 & i \\[3pt]
      0 & 1 & -\frac12 + \frac12i \\[3pt]
      0 & 0 & 0
    \end{bmatrix}
  \end{gather*}
  The last matrix is row-equivalent to $A$.
\end{solution}

\Exercise5 Prove that the following two matrices are {\em not}
row-equivalent:
\begin{equation*}
  \begin{bmatrix}
    2 & 0 & 0 \\
    a & -1 & 0 \\
    b & c & 3
  \end{bmatrix},
  \quad
  \begin{bmatrix}
    1 & 1 & 2 \\
    -2 & 0 & -1 \\
    1 & 3 & 5
  \end{bmatrix}.
\end{equation*}
\begin{proof}
  By performing row operations on the first matrix, we get
  \begin{gather*}
    \begin{bmatrix}
      2 & 0 & 0 \\
      a & -1 & 0 \\
      b & c & 3
    \end{bmatrix}
    \xrightarrow{(1)}
    \begin{bmatrix}
      1 & 0 & 0 \\
      a & -1 & 0 \\
      b & c & 3
    \end{bmatrix}
    \xrightarrow{(2)}
    \begin{bmatrix}
      1 & 0 & 0 \\
      0 & -1 & 0 \\
      0 & c & 3
    \end{bmatrix}
    \xrightarrow{(1)} \\
    \begin{bmatrix}
      1 & 0 & 0 \\
      0 & 1 & 0 \\
      0 & c & 3
    \end{bmatrix}
    \xrightarrow{(2)}
    \begin{bmatrix}
      1 & 0 & 0 \\
      0 & 1 & 0 \\
      0 & 0 & 3
    \end{bmatrix}
    \xrightarrow{(1)}
    \begin{bmatrix}
      1 & 0 & 0 \\
      0 & 1 & 0 \\
      0 & 0 & 1
    \end{bmatrix}.
  \end{gather*}
  We see that this matrix is row-equivalent to the identity
  matrix. The corresponding system of equations has only the trivial
  solution.

  For the second matrix, we get
  \begin{gather*}
    \begin{bmatrix}
      1 & 1 & 2 \\
      -2 & 0 & -1 \\
      1 & 3 & 5
    \end{bmatrix}
    \xrightarrow{(2)}
    \begin{bmatrix}
      1 & 1 & 2 \\
      0 & 2 & 3 \\
      0 & 2 & 3
    \end{bmatrix}
    \xrightarrow{(1)}
    \begin{bmatrix}
      1 & 1 & 2 \\[3pt]
      0 & 1 & \frac32 \\[3pt]
      0 & 2 & 3
    \end{bmatrix}
    \xrightarrow{(2)} \\
    \begin{bmatrix}
      1 & 1 & 2 \\[3pt]
      0 & 1 & \frac32 \\[3pt]
      0 & 0 & 0
    \end{bmatrix}
    \xrightarrow{(2)}
    \begin{bmatrix}
      1 & 0 & \frac12 \\[3pt]
      0 & 1 & \frac32 \\[3pt]
      0 & 0 & 0
    \end{bmatrix}.
  \end{gather*}
  The system of equations corresponding to this matrix has nontrivial
  solutions. Therefore the two matrices are not row-equivalent.
\end{proof}

\Exercise6 Let
\begin{equation*}
  A =
  \begin{bmatrix}
    a & b \\
    c & d
  \end{bmatrix}
\end{equation*}
be a $2\times2$ matrix with complex entries. Suppose that $A$ is
row-reduced and also that $a + b + c + d = 0$. Prove that there are
exactly three such matrices.
\begin{proof}
  One possibility is the zero matrix,
  \begin{equation*}
    A =
    \begin{bmatrix}
      0 & 0 \\
      0 & 0
    \end{bmatrix}.
  \end{equation*}
  If $A$ is not the zero matrix, then it has at least one nonzero
  row. If it has exactly one nonzero row, then in order to satisfy the
  given constraints, the nonzero row will have a $1$ in the first
  column and a $-1$ in the second. This gives two possibilities,
  \begin{equation*}
    A =
    \begin{bmatrix}
      1 & -1 \\
      0 & 0
    \end{bmatrix}
    \quad\text{or}\quad
    A =
    \begin{bmatrix}
      0 & 0 \\
      1 & -1
    \end{bmatrix}.
  \end{equation*}

  Finally, if $A$ has two nonzero rows, then it must be the identity
  matrix or the matrix
  $\left[\begin{smallmatrix} 0 & 1 \\ 1 & 0 \end{smallmatrix}\right]$,
  but neither of these are valid since the sum of the entries is
  nonzero in each case. Thus there are only the three possibilities
  given above.
\end{proof}

\Exercise7 Prove that the interchange of two rows of a matrix can be
accomplished by a finite sequence of elementary row operations of the
other two types.
\begin{proof}
  We can, without loss of generality, assume that the matrix has only
  two rows, since any additional rows could just be ignored in the
  procedure that follows. Let this matrix be given by
  \begin{equation*}
    A_0 =
    \begin{bmatrix}
      a_1 & a_2 & a_3 & \cdots & a_n \\
      b_1 & b_2 & b_3 & \cdots & b_n
    \end{bmatrix}.
  \end{equation*}

  First, add $-1$ times row $2$ to row $1$ to get the matrix
  \begin{equation*}
    A_1 =
    \begin{bmatrix}
      a_1 - b_1 & a_2 - b_2 & a_3 - b_3 & \cdots & a_n - b_n \\
      b_1 & b_2 & b_3 & \cdots & b_n
    \end{bmatrix}.
  \end{equation*}
  Next, add row $1$ to row $2$ to get
  \begin{equation*}
    A_2 =
    \begin{bmatrix}
      a_1 - b_1 & a_2 - b_2 & a_3 - b_3 & \cdots & a_n - b_n \\
      a_1 & a_2 & a_3 & \cdots & a_n
    \end{bmatrix},
  \end{equation*}
  and then add $-1$ times row $2$ to row $1$, which gives
  \begin{equation*}
    A_3 =
    \begin{bmatrix}
      -b_1 & -b_2 & -b_3 & \cdots & -b_n \\
      a_1 & a_2 & a_3 & \cdots & a_n
    \end{bmatrix}.
  \end{equation*}
  For the final step, multiply row $1$ by $-1$ to get
  \begin{equation*}
    A_4 =
    \begin{bmatrix}
      b_1 & b_2 & b_3 & \cdots & b_n \\
      a_1 & a_2 & a_3 & \cdots & a_n
    \end{bmatrix}.
  \end{equation*}
  We can see that $A_4$ has the same entries as $A_0$ but with the
  rows interchanged. And only a finite number of elementary row
  operations of the first two kinds were performed.
\end{proof}
