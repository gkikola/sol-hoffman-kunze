\section{Invertible Matrices}

\Exercise1
\label{eq:lin-eq:inv:R-eq-P-A}
Let
\begin{equation*}
  A =
  \begin{bmatrix}
    1 & 2 & 1 & 0 \\
    -1 & 0 & 3 & 5 \\
    1 & -2 & 1 & 1
  \end{bmatrix}.
\end{equation*}
Find a row-reduced echelon matrix $R$ which is row-equivalent to $A$
and an invertible $3\times3$ matrix $P$ such that $R = PA$.
\begin{solution}
  We can perform elementary row operations on $A$, while performing
  the same operations on $I$, in order to find $R$ and $P$:
  \begin{align*}
    \begin{bmatrix}
      1 & 2 & 1 & 0 \\
      -1 & 0 & 3 & 5 \\
      1 & -2 & 1 & 1
    \end{bmatrix},
    &\quad
    \begin{bmatrix}
      1 & 0 & 0 \\
      0 & 1 & 0 \\
      0 & 0 & 1
    \end{bmatrix} \\
    \begin{bmatrix}
      1 & 2 & 1 & 0 \\
      0 & 2 & 4 & 5 \\
      0 & -4 & 0 & 1
    \end{bmatrix},
    &\quad
    \begin{bmatrix}
      1 & 0 & 0 \\
      1 & 1 & 0 \\
      -1 & 0 & 1
    \end{bmatrix} \\
    \begin{bmatrix}
      1 & 0 & -3 & -5 \\
      0 & 2 & 4 & 5 \\
      0 & 0 & 8 & 11
    \end{bmatrix},
    &\quad
    \begin{bmatrix}
      0 & -1 & 0 \\
      1 & 1 & 0 \\
      1 & 2 & 1
    \end{bmatrix} \\
    \begin{bmatrix}
      1 & 0 & -3 & -5 \\[3pt]
      0 & 1 & 2 & \frac52 \\[3pt]
      0 & 0 & 1 & \frac{11}8
    \end{bmatrix},
    &\quad
    \begin{bmatrix}
      0 & -1 & 0 \\[3pt]
      \frac12 & \frac12 & 0 \\[3pt]
      \frac18 & \frac14 & \frac18
    \end{bmatrix} \\
    \begin{bmatrix}
      1 & 0 & 0 & -\frac78 \\[3pt]
      0 & 1 & 0 & -\frac14 \\[3pt]
      0 & 0 & 1 & \frac{11}8
    \end{bmatrix},
    &\quad
    \begin{bmatrix}
      \frac38 & -\frac14 & \frac38 \\[3pt]
      \frac14 & 0 & -\frac14 \\[3pt]
      \frac18 & \frac14 & \frac18
    \end{bmatrix}.
  \end{align*}
  Therefore,
  \begin{equation*}
    R =
    \begin{bmatrix}
      1 & 0 & 0 & -\frac78 \\[3pt]
      0 & 1 & 0 & -\frac14 \\[3pt]
      0 & 0 & 1 & \frac{11}8
    \end{bmatrix},
    \quad
    P =
    \begin{bmatrix}
      \frac38 & -\frac14 & \frac38 \\[3pt]
      \frac14 & 0 & -\frac14 \\[3pt]
      \frac18 & \frac14 & \frac18
    \end{bmatrix}
    =
    \frac18
    \begin{bmatrix}
      3 & -2 & 3 \\
      2 & 0 & -2 \\
      1 & 2 & 1
    \end{bmatrix},
  \end{equation*}
  and $R = PA$.
\end{solution}

\Exercise2 Do Exercise~\ref{eq:lin-eq:inv:R-eq-P-A}, but with
\begin{equation*}
  A =
  \begin{bmatrix}
    2 & 0 & i \\
    1 & -3 & -i \\
    i & 1 & 1
  \end{bmatrix}.
\end{equation*}
\begin{solution}
  We proceed as before:
  \begin{align*}
    \begin{bmatrix}
      2 & 0 & i \\
      1 & -3 & -i \\
      i & 1 & 1
    \end{bmatrix},
    &\quad
    \begin{bmatrix}
      1 & 0 & 0 \\
      0 & 1 & 0 \\
      0 & 0 & 1
    \end{bmatrix} \\
    \begin{bmatrix}
      1 & -3 & -i \\
      2 & 0 & i \\
      i & 1 & 1
    \end{bmatrix},
    &\quad
    \begin{bmatrix}
      0 & 1 & 0 \\
      1 & 0 & 0 \\
      0 & 0 & 1
    \end{bmatrix} \\
    \begin{bmatrix}
      1 & -3 & -i \\
      0 & 6 & 3i \\
      0 & 1 + 3i & 0
    \end{bmatrix},
    &\quad
    \begin{bmatrix}
      0 & 1 & 0 \\
      1 & -2 & 0 \\
      0 & -i & 1
    \end{bmatrix} \\
    \begin{bmatrix}
      1 & -3 & -i \\[3pt]
      0 & 1 & \frac12i \\[3pt]
      0 & 1 + 3i & 0
    \end{bmatrix},
    &\quad
    \begin{bmatrix}
      0 & 1 & 0 \\[3pt]
      \frac16 & -\frac13 & 0 \\[3pt]
      0 & -i & 1
    \end{bmatrix} \\
    \begin{bmatrix}
      1 & 0 & \frac12i \\[3pt]
      0 & 1 & \frac12i \\[3pt]
      0 & 0 & \frac32 - \frac12i
    \end{bmatrix},
    &\quad
    \begin{bmatrix}
      \frac12 & 0 & 0 \\[3pt]
      \frac16 & -\frac13 & 0 \\[3pt]
      -\frac16 - \frac12i & \frac13 & 1
    \end{bmatrix} \\
    \begin{bmatrix}
      1 & 0 & \frac12i \\[3pt]
      0 & 1 & \frac12i \\[3pt]
      0 & 0 & 1
    \end{bmatrix},
    &\quad
    \begin{bmatrix}
      \frac12 & 0 & 0 \\[3pt]
      \frac16 & -\frac13 & 0 \\[3pt]
      -\frac13i & \frac15 + \frac1{15}i & \frac35 + \frac15i
    \end{bmatrix} \\
    \begin{bmatrix}
      1 & 0 & 0 \\[3pt]
      0 & 1 & 0 \\[3pt]
      0 & 0 & 1
    \end{bmatrix},
    &\quad
    \begin{bmatrix}
      \frac13 & \frac1{30}-\frac1{10}i & \frac1{10}-\frac3{10}i \\[3pt]
      0 & -\frac3{10}-\frac1{10}i & \frac1{10} - \frac3{10}i \\[3pt]
      -\frac13i & \frac15 + \frac1{15}i & \frac35 + \frac15i
    \end{bmatrix}.
  \end{align*}
  So
  \begin{equation*}
    R =
    \begin{bmatrix}
      1 & 0 & 0 \\
      0 & 1 & 0 \\
      0 & 0 & 1
    \end{bmatrix}
    = I,
    \quad
    P = \frac1{30}
    \begin{bmatrix}
      10 & 1 - 3i & 3 - 9i \\
      0 & -9 - 3i & 3 - 9i \\
      -10i & 6 + 2i & 18 + 6i
    \end{bmatrix},
  \end{equation*}
  and $R = PA$.
\end{solution}

\Exercise3 For each of the two matrices
\begin{equation*}
  \begin{bmatrix}
    2 & 5 & -1 \\
    4 & -1 & 2 \\
    6 & 4 & 1
  \end{bmatrix},
  \quad
  \begin{bmatrix}
    1 & -1 & 2 \\
    3 & 2 & 4 \\
    0 & 1 & -2
  \end{bmatrix}
\end{equation*}
use elementary row operations to discover whether it is invertible,
and to find the inverse in case it is.
\begin{solution}
  For the first matrix, row-reduction gives
  \begin{gather*}
    \begin{bmatrix}
      2 & 5 & -1 \\
      4 & -1 & 2 \\
      6 & 4 & 1
    \end{bmatrix}
    \xrightarrow{(1)}
    \begin{bmatrix}
      1 & \frac52 & -\frac12 \\[3pt]
      4 & -1 & 2 \\[3pt]
      6 & 4 & 1
    \end{bmatrix}
    \xrightarrow{(2)}
    \begin{bmatrix}
      1 & \frac52 & -\frac12 \\[3pt]
      0 & -11 & 4 \\[3pt]
      0 & -11 & 4
    \end{bmatrix}
    \xrightarrow{(2)}
    \begin{bmatrix}
      1 & \frac52 & -\frac12 \\[3pt]
      0 & -11 & 4 \\[3pt]
      0 & 0 & 0
    \end{bmatrix},
  \end{gather*}
  and we see that the original matrix is not invertible since it is
  row-equivalent to a matrix having a row of zeros.

  For the second matrix, we get
  \begin{align*}
    \begin{bmatrix}
      1 & -1 & 2 \\
      3 & 2 & 4 \\
      0 & 1 & -2
    \end{bmatrix},
    &\quad
    \begin{bmatrix}
      1 & 0 & 0 \\
      0 & 1 & 0 \\
      0 & 0 & 1
    \end{bmatrix} \\
    \begin{bmatrix}
      1 & -1 & 2 \\
      0 & 5 & -2 \\
      0 & 1 & -2
    \end{bmatrix},
    &\quad
    \begin{bmatrix}
      1 & 0 & 0 \\
      -3 & 1 & 0 \\
      0 & 0 & 1
    \end{bmatrix} \\
    \begin{bmatrix}
      1 & -1 & 2 \\
      0 & 1 & -2 \\
      0 & 5 & -2
    \end{bmatrix},
    &\quad
    \begin{bmatrix}
      1 & 0 & 0 \\
      0 & 0 & 1 \\
      -3 & 1 & 0
    \end{bmatrix} \\
    \begin{bmatrix}
      1 & 0 & 0 \\
      0 & 1 & -2 \\
      0 & 0 & 8
    \end{bmatrix},
    &\quad
    \begin{bmatrix}
      1 & 0 & 1 \\
      0 & 0 & 1 \\
      -3 & 1 & -5
    \end{bmatrix} \\
    \begin{bmatrix}
      1 & 0 & 0 \\[3pt]
      0 & 1 & -2 \\[3pt]
      0 & 0 & 1
    \end{bmatrix},
    &\quad
    \begin{bmatrix}
      1 & 0 & 1 \\[3pt]
      0 & 0 & 1 \\[3pt]
      -\frac38 & \frac18 & -\frac58
    \end{bmatrix} \\
    \begin{bmatrix}
      1 & 0 & 0 \\[3pt]
      0 & 1 & 0 \\[3pt]
      0 & 0 & 1
    \end{bmatrix},
    &\quad
    \begin{bmatrix}
      1 & 0 & 1 \\[3pt]
      -\frac34 & \frac14 & -\frac14 \\[3pt]
      -\frac38 & \frac18 & -\frac58
    \end{bmatrix}.
  \end{align*}
  From this we see that the original matrix is invertible and its
  inverse is the matrix
  \begin{equation*}
    \frac18
    \begin{bmatrix}
      8 & 0 & 8 \\
      -6 & 2 & -2 \\
      -3 & 1 & -5
    \end{bmatrix}. \qedhere
  \end{equation*}
\end{solution}

\Exercise4 Let
\begin{equation*}
  A =
  \begin{bmatrix}
    5 & 0 & 0 \\
    1 & 5 & 0 \\
    0 & 1 & 5
  \end{bmatrix}.
\end{equation*}
For which $X$ does there exist a scalar $c$ such that $AX = cX$?
\begin{solution}
  Let
  \begin{equation*}
    X =
    \begin{bmatrix}
      x_1 \\
      x_2 \\
      x_3
    \end{bmatrix}.
  \end{equation*}
  Then $AX = cX$ implies
  \begin{align*}
    5x_1 &= cx_1 \\
    x_1 + 5x_2 &= cx_2 \\
    x_2 + 5x_3 &= cx_3,
  \end{align*}
  and this is a homogeneous system of equations with coefficient
  matrix
  \begin{equation*}
    B =
    \begin{bmatrix}
      5 - c & 0 & 0 \\
      1 & 5 - c & 0 \\
      0 & 1 & 5 - c
    \end{bmatrix}.
  \end{equation*}
  If $c = 5$ then $(x_1,x_2,x_3) = (0,0,t)$ for some scalar $t$, so
  this gives one possibility for $X$. If we assume $c\neq5$, then the
  matrix $B$ can be row-reduced to the identity matrix, so that
  $X = 0$ is then the only possibility. Therefore, there is a scalar
  $c$ with $AX = cX$ if and only if
  \begin{equation*}
    X =
    \begin{bmatrix}
      0 \\
      0 \\
      t
    \end{bmatrix},
  \end{equation*}
  for some arbitrary scalar $t$.
\end{solution}
