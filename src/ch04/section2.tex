\chapter{Polynomials}

\skipsection
\section{The Algebra of Polynomials}

\Exercise1 Let $F$ be a subfield of the complex numbers and let $A$ be
the following $2\times2$ matrix over $F$,
\begin{equation*}
  A =
  \begin{bmatrix}
    2 & 1 \\
    -1 & 3
  \end{bmatrix}.
\end{equation*}
For each of the following polynomials $f$ over $F$, compute $f(A)$.
\begin{enumerate}
\item $f = x^2 - x + 2$
  \begin{solution}
    We have
    \begin{align*}
      f(A)
      &=
      \begin{bmatrix}
        2 & 1 \\
        -1 & 3
      \end{bmatrix}^2
      -
      \begin{bmatrix}
        2 & 1 \\
        -1 & 3
      \end{bmatrix}
      + 2
      \begin{bmatrix}
        1 & 0 \\
        0 & 1
      \end{bmatrix} \\
      &=
      \begin{bmatrix}
        3 & 5 \\
        -5 & 8
      \end{bmatrix}
      -
      \begin{bmatrix}
        2 & 1 \\
        -1 & 3
      \end{bmatrix}
      +
      \begin{bmatrix}
        2 & 0 \\
        0 & 2
      \end{bmatrix} \\
      &=
      \begin{bmatrix}
        3 & 4 \\
        -4 & 7
      \end{bmatrix}. \qedhere
    \end{align*}
  \end{solution}

\item $f = x^3 - 1$
  \begin{solution}
    Since
    \begin{equation*}
      A^3 =
      \begin{bmatrix}
        1 & 18 \\
        -18 & 19
      \end{bmatrix},
    \end{equation*}
    we get
    \begin{equation*}
      f(A) = A^3 - I =
      \begin{bmatrix}
        0 & 18 \\
        -18 & 18
      \end{bmatrix}. \qedhere
    \end{equation*}
  \end{solution}

\item $f = x^2 - 5x + 7$
  \begin{solution}
    In this case, we have
    \begin{align*}
      f(A)
      &=
      \begin{bmatrix}
        2 & 1 \\
        -1 & 3
      \end{bmatrix}^2
      - 5
      \begin{bmatrix}
        2 & 1 \\
        -1 & 3
      \end{bmatrix}
      + 7
      \begin{bmatrix}
        1 & 0 \\
        0 & 1
      \end{bmatrix} \\
      &=
      \begin{bmatrix}
        3 & 5 \\
        -5 & 8
      \end{bmatrix}
      +
      \begin{bmatrix}
        -10 & -5 \\
        5 & -15
      \end{bmatrix}
      +
      \begin{bmatrix}
        7 & 0 \\
        0 & 7
      \end{bmatrix} \\
      &=
      \begin{bmatrix}
        0 & 0 \\
        0 & 0
      \end{bmatrix}. \qedhere
    \end{align*}
  \end{solution}
\end{enumerate}

\Exercise2 Let $T$ be the linear operator on $R^3$ defined by
\begin{equation*}
  T(x_1, x_2, x_3) = (x_1, x_3, -2x_2 - x_3).
\end{equation*}
Let $f$ be the polynomial over $R$ defined by $f = -x^3 + 2$. Find
$f(T)$.
\begin{solution}
  $f(T) = -T^3 + 2I$, where $I$ denotes the identity transformation on
  $R^3$. We compute,
  \begin{align*}
    f(T)(x_1,x_2,x_3)
    &= (-T^3)(x_1,x_2,x_3) + 2(x_1,x_2,x_3) \\
    &= -T^2(x_1, x_3, -2x_2 - x_3) + 2(x_1,x_2,x_3) \\
    &= -T(x_1, -2x_2 - x_3, 2x_2 - x_3) + (2x_1,2x_2,2x_3) \\
    &= -(x_1, 2x_2 - x_3, 2x_2 + 3x_3) + (2x_1,2x_2,2x_3) \\
    &= (x_1, x_3, -2x_2 - x_3),
  \end{align*}
  and we see that $f(T) = T$.
\end{solution}

\Exercise3 Let $A$ be an $n\times n$ diagonal matrix over the field
$F$, i.e., a matrix satisfying $A_{ij} = 0$ for $i\neq j$. Let $f$ be
the polynomial over $F$ defined by
\begin{equation*}
  f = (x - A_{11})\cdots(x - A_{nn}).
\end{equation*}
What is the matrix $f(A)$?
\begin{solution}
  For each $i$ with $1\leq i\leq n$, define the polynomial
  \begin{equation*}
    f_i = x - A_{ii},
  \end{equation*}
  so that $f = f_1f_2\cdots f_n$. Observe that the matrix $f_i(A)$
  is a diagonal matrix with
  \begin{equation*}
    (f_i(A))_{jj} = A_{jj} - A_{ii}.
  \end{equation*}

  Now consider the matrix
  \begin{equation*}
    B = f_1(A)f_2(A)\cdots f_n(A).
  \end{equation*}
  By Theorem~2, we must have $f(A) = B$. To describe $f(A)$, we need
  only consider the entries of $B$. Observe that the product of two
  (or more) diagonal matrices must also be diagonal. Moreover, the
  entries along the diagonal of the product will simply be the product
  of the corresponding entries in the original matrices. So for each
  $i$ with $1\leq i\leq n$, we get
  \begin{equation*}
    B_{ii} = (A_{ii} - A_{11})(A_{ii} - A_{22})\cdots
    (A_{ii} - A_{ii})\cdots(A_{ii} - A_{nn}) = 0.
  \end{equation*}
  Since $B$ is a diagonal matrix with zeroes on the diagonal, we
  simply have $B = 0$. In other words, $f(A)$ is the $n\times n$ zero
  matrix.
\end{solution}
