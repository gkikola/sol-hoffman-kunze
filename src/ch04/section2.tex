\chapter{Polynomials}

\skipsection
\section{The Algebra of Polynomials}

\Exercise1 Let $F$ be a subfield of the complex numbers and let $A$ be
the following $2\times2$ matrix over $F$,
\begin{equation*}
  A =
  \begin{bmatrix}
    2 & 1 \\
    -1 & 3
  \end{bmatrix}.
\end{equation*}
For each of the following polynomials $f$ over $F$, compute $f(A)$.
\begin{enumerate}
\item $f = x^2 - x + 2$
  \begin{solution}
    We have
    \begin{align*}
      f(A)
      &=
      \begin{bmatrix}
        2 & 1 \\
        -1 & 3
      \end{bmatrix}^2
      -
      \begin{bmatrix}
        2 & 1 \\
        -1 & 3
      \end{bmatrix}
      + 2
      \begin{bmatrix}
        1 & 0 \\
        0 & 1
      \end{bmatrix} \\
      &=
      \begin{bmatrix}
        3 & 5 \\
        -5 & 8
      \end{bmatrix}
      -
      \begin{bmatrix}
        2 & 1 \\
        -1 & 3
      \end{bmatrix}
      +
      \begin{bmatrix}
        2 & 0 \\
        0 & 2
      \end{bmatrix} \\
      &=
      \begin{bmatrix}
        3 & 4 \\
        -4 & 7
      \end{bmatrix}. \qedhere
    \end{align*}
  \end{solution}

\item $f = x^3 - 1$
  \begin{solution}
    Since
    \begin{equation*}
      A^3 =
      \begin{bmatrix}
        1 & 18 \\
        -18 & 19
      \end{bmatrix},
    \end{equation*}
    we get
    \begin{equation*}
      f(A) = A^3 - I =
      \begin{bmatrix}
        0 & 18 \\
        -18 & 18
      \end{bmatrix}. \qedhere
    \end{equation*}
  \end{solution}

\item $f = x^2 - 5x + 7$
  \begin{solution}
    In this case, we have
    \begin{align*}
      f(A)
      &=
      \begin{bmatrix}
        2 & 1 \\
        -1 & 3
      \end{bmatrix}^2
      - 5
      \begin{bmatrix}
        2 & 1 \\
        -1 & 3
      \end{bmatrix}
      + 7
      \begin{bmatrix}
        1 & 0 \\
        0 & 1
      \end{bmatrix} \\
      &=
      \begin{bmatrix}
        3 & 5 \\
        -5 & 8
      \end{bmatrix}
      +
      \begin{bmatrix}
        -10 & -5 \\
        5 & -15
      \end{bmatrix}
      +
      \begin{bmatrix}
        7 & 0 \\
        0 & 7
      \end{bmatrix} \\
      &=
      \begin{bmatrix}
        0 & 0 \\
        0 & 0
      \end{bmatrix}. \qedhere
    \end{align*}
  \end{solution}
\end{enumerate}

\Exercise2 Let $T$ be the linear operator on $R^3$ defined by
\begin{equation*}
  T(x_1, x_2, x_3) = (x_1, x_3, -2x_2 - x_3).
\end{equation*}
Let $f$ be the polynomial over $R$ defined by $f = -x^3 + 2$. Find
$f(T)$.
\begin{solution}
  $f(T) = -T^3 + 2I$, where $I$ denotes the identity transformation on
  $R^3$. We compute,
  \begin{align*}
    f(T)(x_1,x_2,x_3)
    &= (-T^3)(x_1,x_2,x_3) + 2(x_1,x_2,x_3) \\
    &= -T^2(x_1, x_3, -2x_2 - x_3) + 2(x_1,x_2,x_3) \\
    &= -T(x_1, -2x_2 - x_3, 2x_2 - x_3) + (2x_1,2x_2,2x_3) \\
    &= -(x_1, 2x_2 - x_3, 2x_2 + 3x_3) + (2x_1,2x_2,2x_3) \\
    &= (x_1, x_3, -2x_2 - x_3),
  \end{align*}
  and we see that $f(T) = T$.
\end{solution}

\Exercise3 Let $A$ be an $n\times n$ diagonal matrix over the field
$F$, i.e., a matrix satisfying $A_{ij} = 0$ for $i\neq j$. Let $f$ be
the polynomial over $F$ defined by
\begin{equation*}
  f = (x - A_{11})\cdots(x - A_{nn}).
\end{equation*}
What is the matrix $f(A)$?
\begin{solution}
  For each $i$ with $1\leq i\leq n$, define the polynomial
  \begin{equation*}
    f_i = x - A_{ii},
  \end{equation*}
  so that $f = f_1f_2\cdots f_n$. Observe that the matrix $f_i(A)$
  is a diagonal matrix with
  \begin{equation*}
    (f_i(A))_{jj} = A_{jj} - A_{ii}.
  \end{equation*}

  Now consider the matrix
  \begin{equation*}
    B = f_1(A)f_2(A)\cdots f_n(A).
  \end{equation*}
  By Theorem~2, we must have $f(A) = B$. To describe $f(A)$, we need
  only consider the entries of $B$. Observe that the product of two
  (or more) diagonal matrices must also be diagonal. Moreover, the
  entries along the diagonal of the product will simply be the product
  of the corresponding entries in the original matrices. So for each
  $i$ with $1\leq i\leq n$, we get
  \begin{equation*}
    B_{ii} = (A_{ii} - A_{11})(A_{ii} - A_{22})\cdots
    (A_{ii} - A_{ii})\cdots(A_{ii} - A_{nn}) = 0.
  \end{equation*}
  Since $B$ is a diagonal matrix with zeroes on the diagonal, we
  simply have $B = 0$. In other words, $f(A)$ is the $n\times n$ zero
  matrix.
\end{solution}

\Exercise4 If $f$ and $g$ are independent polynomials over a field $F$
and $h$ is a non-zero polynomial over $F$, show that $fh$ and $gh$ are
independent.
\begin{proof}
  Let $a$ and $b$ be scalars in $F$ such that
  \begin{equation*}
    a(fh) + b(gh) = 0.
  \end{equation*}
  Then
  \begin{equation*}
    (af + bg)h = 0,
  \end{equation*}
  and since $h$ is nonzero, it follows from part (i) of Theorem~1 that
  $af+bg = 0$. But $f$ and $g$ are independent, so we must have
  $a = b = 0$. This shows that $fh$ and $gh$ are independent as well.
\end{proof}

\Exercise5 If $F$ is a field, show that the product of two non-zero
elements of $F^\infty$ is non-zero.
\begin{proof}
  Suppose the contrary, and let $p$ and $q$ be nonzero elements of
  $F^\infty$ such that $pq = 0$. Let $m$ be smallest nonnegative
  integer with $p_m\neq0$, and let $n$ be the smallest nonnegative
  integer with $q_n\neq0$. Then
  \begin{equation}
    \label{eq:polynomials:m-plus-n-term-of-product-in-F-inf}
    (pq)_{m+n}
    = \sum_{i=0}^{m+n}p_iq_{m+n-i}.
  \end{equation}
  In the sum on the right-hand side, notice that $p_i = 0$ when
  $i < m$. Notice also that $q_{m+n-i} = 0$ when $m+n-i < n$, i.e.,
  when $i > m$. Consequently,
  equation~\eqref{eq:polynomials:m-plus-n-term-of-product-in-F-inf}
  becomes
  \begin{equation*}
    (pq)_{m+n} = p_mq_n.
  \end{equation*}
  But then $(pq)_{m+n}\neq0$, contradicting the assumption that
  $pq = 0$. Therefore the product of non-zero elements in $F^\infty$
  cannot be zero.
\end{proof}

\Exercise6 Let $S$ be a set of non-zero polynomials over a field
$F$. If no two elements of $S$ have the same degree, show that $S$ is
an independent set in $F[x]$.
\begin{proof}
  Let $S$ be as stated, let $p_1,\dots,p_n$ be polynomials in $S$, and
  let $c_1,\dots,c_n$ be scalars in $F$ such that
  \begin{equation}
    \label{eq:polynomials:lin-comb-diff-degrees}
    \sum_{i=1}^nc_ip_i = 0.
  \end{equation}
  We want to show that $c_i = 0$ for each $i$. Since
  $\{p_1, \dots, p_n\}$ is a finite set of polynomials, no two of
  which have the same degree, there must be a polynomial in the set
  with a degree that is strictly larger than the degrees of all of the
  other polynomials. Without loss of generality, we may suppose that
  this polynomial of maximum degree is $p_n$. If $c_n\neq0$, then we
  may write
  \begin{equation*}
    p_n = -\frac1{c_n}\sum_{i=1}^{n-1}c_ip_i.
  \end{equation*}
  By Theorem~1, we then know that the degree of $p_n$ is less than or
  equal to the maximum degree of the remaining polynomials. But this
  is impossible because $p_n$ has a larger degree than
  the others. This contradiction shows that $c_n = 0$.

  Since $c_n$ is zero, equation
  \eqref{eq:polynomials:lin-comb-diff-degrees} now becomes
  \begin{equation*}
    \sum_{i=1}^{n-1}c_ip_i = 0.
  \end{equation*}
  By the same argument as before, there must be a polynomial among
  $p_1,\dots,p_{n-1}$ with largest degree. Again, no generality is
  lost in assuming it is $p_{n-1}$. Then we may again use the same
  method as before to show that $c_{n-1} = 0$.

  Continuing in this way, we see that the only possibility is that
  \begin{equation*}
    c_1 = c_2 = \cdots = c_n = 0
  \end{equation*}
  and the set $S$ must be independent.
\end{proof}

\Exercise7 If $a$ and $b$ are elements of a field $F$ and $a\neq0$,
show that the polynomials $1$, $ax+b$, $(ax+b)^2$, $(ax+b)^3$, $\dots$
form a basis of $F[x]$.
\begin{proof}
  Call this set of polynomials $S$. We know by the previous exercise
  that $S$ is independent, so we need only show that it spans
  $F[x]$.

  Note that the polynomial $1$ is clearly in the span of $S$. And the
  polynomial $x$ is in the span of $S$ since
  \begin{equation*}
    x = -\frac{b}a + \frac1a(ax+b).
  \end{equation*}
  Now suppose that $x^i$ is in the span of $S$. Since
  \begin{equation*}
    x^{i+1} = x\cdot x^i,
  \end{equation*}
  we see that $x^{i+1}$ is also in the span of $S$. By induction, each
  of the polynomials $1$, $x$, $x^2$, $x^3$, $\dots$ is in the span of
  $S$. But these polynomials span $F[x]$, so $S$ must also span
  $F[x]$.
\end{proof}

\Exercise8 If $F$ is a field and $h$ is a polynomial over $F$ of
degree $\geq1$, show that the mapping $f\to f(h)$ is a one-one linear
transformation of $F[x]$ into $F[x]$. Show that this transformation is
an isomorphism of $F[x]$ onto $F[x]$ if and only if $\deg h = 1$.
\begin{proof}
  Let this map be denoted by $T$, so that $T(f) = f(h)$. For any
  scalar $c$ in $F$ and for any polynomials $f,g$ in $F[x]$, we have
  by Theorem~2 that
  \begin{align*}
    T(cf + g)
    &= (cf + g)(h) \\
    &= (cf)(h) + g(h) \\
    &= cf(h) + g(h) \\
    &= cT(f) + T(g),
  \end{align*}
  so $T$ is a linear transformation.

  Now we want to show that $T$ is one-to-one. Let $f$ and $g$ be
  polynomials in $F[x]$ such that $T(f) = T(g)$. Write
  \begin{equation*}
    f = \sum_{i=0}^ma_ix^i \quad\text{and}\quad g = \sum_{i=0}^nb_ix^i.
  \end{equation*}
  Then
  \begin{equation*}
    T(f) = f(h) = \sum_{i=0}^ma_ih^i
    \quad\text{and}\quad
    T(g) = g(h) = \sum_{i=0}^nb_ih^i.
  \end{equation*}
  Since $T(f) = T(g)$ and since $h$ has degree $\geq1$, we may compare
  the coefficients for each term in the two sums above to see that
  $m = n$ and $a_i = b_i$ for each $i$ with $1\leq i\leq n$. That is,
  $f = g$ and $T$ is one-to-one.

  Lastly, suppose $\deg h = n$, where $n > 1$. We can write
  \begin{equation*}
    h = cx^n + k,
  \end{equation*}
  where $c$ is in $F$ and $k$ is a polynomial of degree less than
  $h$. Then, if
  \begin{equation*}
    f = \sum_{i=0}^ma_ix^i
  \end{equation*}
  is any polynomial in $F[x]$ having degree $m$, we have
  \begin{align*}
    T(f) = f(h)
    &= \sum_{i=0}^ma_ih^i \\
    &= \sum_{i=0}^ma_i(cx^n + k)^i \\
    &= a_mc^mx^{mn} + \ell,
  \end{align*}
  where $\ell$ is a polynomial having degree less than
  $mn$. Consequently, we see that $T(f)$ has a degree that is a
  multiple of $n$. In particular, there is no $f$ such that
  $T(f) = x$, since $n$ does not divide $1$. Therefore $T$ cannot be
  onto. On the other hand, if $\deg h = 1$, then we know that $T$ is
  onto by the previous exercise.
\end{proof}
