\section{The Double Dual}

\Exercise1 Let $n$ be a positive integer and $F$ a field. Let $W$ be
the set of all vectors $(x_1,\dots,x_n)$ in $F^n$ such that
$x_1+\cdots+x_n=0$.
\begin{enumerate}
\item
  \label{itm:lin-tran:W0-f-eql-coef}
  Prove that $W^0$ consists of all linear functionals $f$ of the
  form
  \begin{equation*}
    f(x_1,\dots,x_n) = c\sum_{j=1}^nx_j.
  \end{equation*}
  \begin{proof}
    If $n = 1$, then $W$ is the zero subspace and the result is
    trivial, so we will suppose $n > 1$.

    Let $f$ be in $W^0$. We can find scalars $c_1,\dots,c_n$ in $F$
    such that
    \begin{equation*}
      f(x_1,\dots,x_n) = c_1x_1 + \cdots + c_nx_n.
    \end{equation*}
    We want to show that $c_1 = c_2 = \cdots = c_n$.

    Since $f$ annihilates $W$, in particular we know that
    \begin{equation*}
      f(1, -1, 0, 0, \dots, 0) = c_1 - c_2 = 0.
    \end{equation*}
    Therefore $c_1 = c_2$. Likewise, we know that
    \begin{equation*}
      f(0, 1, -1, 0, 0, \dots, 0) = c_2 - c_3 = 0,
    \end{equation*}
    so $c_2 = c_3$. Continuing in this way, we see that the $c_i$ must
    all be identical. Thus $f$ has the form that was specified.
  \end{proof}

\item Show that the dual space $W^*$ of $W$ can be `naturally'
  identified with the linear functionals
  \begin{equation*}
    f(x_1,\dots,x_n) = c_1x_1 + \cdots + c_nx_n
  \end{equation*}
  on $F^n$ which satisfy $c_1 + \cdots + c_n = 0$.
  \begin{proof}
    Let $U$ be the space of linear functionals
    \begin{equation*}
      f(x_1,\dots,x_n) = c_1x_1 + \cdots + c_nx_n
    \end{equation*}
    on $F^n$ which satisfy $c_1 + \cdots + c_n = 0$. We will show that
    $U$ is isomorphic to $W^*$.

    Let $T$ be the function from $U$ into $W^*$ such that $T(f)$ is
    the restriction of $f$ to $W$. Then $T$ is a linear
    transformation. We will show that it is non-singular. Suppose
    $T(f) = 0$. Then $f(\alpha) = 0$ for all $\alpha$ in $W$, hence
    $f$ belongs to $W^0$. By the result from part
    \ref{itm:lin-tran:W0-f-eql-coef}, we know that $f$ has the form
    \begin{equation*}
      f(x_1,\dots,x_n) = c\sum_{j=1}^nx_j.
    \end{equation*}
    But $f$ belongs to $U$, so $c$ must be $0$ and $f$ is the zero
    functional. That is, we have shown that $T(f) = 0$ implies
    $f = 0$, so $T$ is non-singular and hence one-to-one.

    Now, it can be shown that $U$ and $W$ both have dimension
    $n-1$. For example, if $\{\epsilon_1,\dots,\epsilon_n\}$ is the
    standard ordered basis for $F^n$, then
    \begin{equation*}
      \{\epsilon_1 - \epsilon_i \mid 2\leq i\leq n\}
    \end{equation*}
    is a set of $n - 1$ linearly independent vectors which span
    $W$. We can find a similar basis for $U$. Since $T$ is a
    one-to-one linear transformation between vector spaces of equal
    dimension, $T$ must be invertible and thus is an isomorphism.
  \end{proof}
\end{enumerate}

% \Exercise2 Use Theorem~20 to prove the following. If $W$ is a subspace
% of a finite-dimensional vector space $V$ and if $\{g_1,\dots,g_r\}$ is
% any basis for $W^0$, then
% \begin{equation*}
%   W = \bigcap_{i=1}^rN_{g_i}.
% \end{equation*}
% \begin{proof}
%   Let
%   \begin{equation*}
%     N = \bigcap_{i=1}^rN_{g_i}.
%   \end{equation*}
%   We want to show that $W = N$.

%   First, let $\alpha$ be in $W$. Then $g_i(\alpha) = 0$ for each $i$,
%   so $\alpha$ belongs to $N_{g_i}$ for each $i$. That is, $\alpha$ is
%   in $N$, which shows that $W$ is a subset of $N$.
% \end{proof}

% \Exercise3 Let $S$ be a set, $F$ a field, and $V(S; F)$ the space of
% all functions from $S$ into $F$:
% \begin{align*}
%   (f+g)(x) &= f(x) + g(x) \\
%   (cf)(x) &= cf(x).
% \end{align*}
% Let $W$ be any $n$-dimensional subspace of $V(S; F)$. Show that there
% exist points $x_1,\dots,x_n$ in $S$ and functions $f_1,\dots,f_n$ in
% $W$ such that $f_i(x_j) = \delta_{ij}$.
