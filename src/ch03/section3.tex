\section{Isomorphism}

\Exercise1 Let $V$ be the set of complex numbers and let $F$ be the
field of real numbers. With the usual operations, $V$ is a vector
space over $F$. Describe explicitly an isomorphism of this space onto
$R^2$.
\begin{solution}
  Define the map $T$ from $V$ into $R^2$ by
  \begin{equation*}
    T(a + bi) = (a, b), \quad \text{where $a$ and $b$ belong to $R$.}
  \end{equation*}
  Then for any $c$ in $R$,
  \begin{align*}
    T(c(a_1 + b_1i) + (a_2 + b_2i))
    &= T((ca_1 + a_2) + (cb_1 + b_2)i) \\
    &= (ca_1 + a_2, cb_1 + b_2) \\
    &= c(a_1, b_1) + (a_2, b_2) \\
    &= cT(a_1 + b_1i) + T(a_2 + b_2i),
  \end{align*}
  so $T$ is a linear transformation. It is one to one, since
  \begin{equation*}
    (a,b) = (c,d) \quad\text{implies}\quad
    a + bi = c + di,
  \end{equation*}
  and it is onto since $(a, b)$ is evidently in the range of $T$ for
  all $a,b$ in $R$. Therefore $T$ is an isomorphism and $V$ and $R^2$
  are isomorphic.
\end{solution}

\Exercise2 Let $V$ be a vector space over the field of complex
numbers, and suppose there is an isomorphism $T$ of $V$ onto
$C^3$. Let $\alpha_1$, $\alpha_2$, $\alpha_3$, $\alpha_4$ be vectors
in $V$ such that
\begin{align*}
  T\alpha_1 = (1, 0, i) \quad T\alpha_2 = (-2, 1 + i, 0), \\
  T\alpha_3 = (-1, 1, 1), \quad T\alpha_4 = (\sqrt2, i, 3).
\end{align*}
\begin{enumerate}
\item Is $\alpha_1$ in the subspace spanned by $\alpha_2$ and
  $\alpha_3$?
  \begin{solution}
    If $T\alpha_1$ is in the subspace of $C^3$ spanned by $T\alpha_2$
    and $T\alpha_3$, then there is $x_1,x_2$ in $C$ with
    \begin{alignat*}{3}
      -2x_1 &{}-{}& x_2 &{}={}& 1 \\
      (1 + i)x_1 &{}+{}& x_2 &{}={}& 0 \\
      && x_2 &{}={}& i.
    \end{alignat*}
    We reduce the augmented matrix for this system of equations to get
    \begin{equation*}
      \begin{bmatrix}
        -2 & -1 & 1 \\
        1 + i & 1 & 0 \\
        0 & 1 & i
      \end{bmatrix}
      \rightarrow
      \begin{bmatrix}
        1 & 0 & -\frac12 - \frac12i \\[3pt]
        0 & 1 & i \\[3pt]
        0 & 0 & 0
      \end{bmatrix},
    \end{equation*}
    so
    \begin{equation*}
      T\alpha_1 = \left(-\frac12 - \frac12i\right)T\alpha_2 + iT\alpha_3.
    \end{equation*}
    Since $T$ is an isomorphism, it follows that
    \begin{equation*}
      T\alpha_1
      = T\left(\left(-\frac12 - \frac12i\right)\alpha_2 + i\alpha_3\right)
    \end{equation*}
    or
    \begin{equation*}
      \alpha_1 = \left(-\frac12 - \frac12i\right)\alpha_2 + i\alpha_3.
    \end{equation*}
    Therefore $\alpha_1$ is in the subspace spanned by $\alpha_2$ and
    $\alpha_3$.
  \end{solution}

\item Let $W_1$ be the subspace spanned by $\alpha_1$ and $\alpha_2$,
  and let $W_2$ be the subspace spanned by $\alpha_3$ and
  $\alpha_4$. What is the intersection of $W_1$ and $W_2$?
  \begin{solution}
    Let $\alpha$ be in $W_1\cap W_2$. Then $\alpha$ is a linear
    combination of $\alpha_1$ and $\alpha_2$, and also a linear
    combination of $\alpha_3$ and $\alpha_4$. We can therefore find
    $c_1$, $c_2$, $c_3$, and $c_4$ in $C$ with
    \begin{equation*}
      \alpha = c_1\alpha_1 + c_2\alpha_2 = c_3\alpha_3 + c_4\alpha_4.
    \end{equation*}
    So
    \begin{equation*}
      T(c_1\alpha_1 + c_2\alpha_2 - c_3\alpha_3 - c_4\alpha_4) = 0
    \end{equation*}
    which implies that
    \begin{equation*}
      c_1T\alpha_1 + c_2T\alpha_2 - c_3T\alpha_3 - c_4T\alpha_4 = 0.
    \end{equation*}
    This then leads to the system of equations
    \begin{alignat*}{5}
      c_1 &{}-{}& 2c_2 &{}+{}& c_3 &{}-{}& \sqrt2c_4 &{}={}& 0 & \\
      && (1 + i)c_2 &{}-{}& c_3 &{}-{}& ic_4 &{}={}& 0 & \\
      ic_1 && &{}-{}& c_3 &{}-{}& 3c_4 &{}={}& 0 &.
    \end{alignat*}
    The coefficient matrix for this system reduces to
    \begin{equation*}
      \begin{bmatrix}
        1 & -2 & 1 & -\sqrt2 \\
        0 & 1 + i & -1 & -i \\
        i & 0 & -1 & -3
      \end{bmatrix}
      \rightarrow
      \begin{bmatrix}
        1 & 0 & i & 0 \\[3pt]
        0 & 1 & -\frac12 + \frac12i & 0 \\[3pt]
        0 & 0 & 0 & 1
      \end{bmatrix}.
    \end{equation*}
    So, letting $t = -2c_3$, we get
    \begin{equation*}
      2it\alpha_1 + (i - 1)t\alpha_2 + 2t\alpha_3 = 0.
    \end{equation*}
    In particular, $c_4 = 0$ and we see that $\alpha = -2t\alpha_3$
    where $t$ is arbitrary. The space $W_1\cap W_2$ therefore has
    $\{\alpha_3\}$ as a basis. Consequently, this space is the
    one-dimensional subspace consisting of scalar multiples of
    $\alpha_3$.
  \end{solution}

\item Find a basis for the subspace of $V$ spanned by the four vectors
  $\alpha_j$.
  \begin{solution}
    We have already seen previously that $\alpha_3$ can be written as
    a linear combination of $\alpha_1$ and $\alpha_2$:
    \begin{equation*}
      \alpha_3 = -i\alpha_1 + \frac{1 - i}2\alpha_2.
    \end{equation*}
    A check will show that the remaining vectors are linearly
    independent. Therefore $\{\alpha_1,\alpha_2,\alpha_4\}$ forms a
    basis for the subspace of $V$ spanned by $\alpha_j$.
  \end{solution}
\end{enumerate}

\Exercise3 Let $W$ be the set of all $2\times2$ complex Hermitian
matrices, that is, the set of $2\times2$ complex matrices $A$ such
that $A_{ij} = \overline{A_{ji}}$ (the bar denoting complex
conjugation). As we pointed out in Example~6 of Chapter~2, $W$ is a
vector space over the field of {\em real} numbers, under the usual
operations. Verify that
\begin{equation*}
  (x,y,z,t)\rightarrow
  \begin{bmatrix}
    t + x & y + iz \\
    y - iz & t - x
  \end{bmatrix}
\end{equation*}
is an isomorphism of $R^4$ onto $W$.
\begin{proof}
  Denote this mapping by $T$. Then for any
  \begin{equation*}
    \alpha = (x_1,y_1,z_1,t_1) \quad\text{and}\quad
    \beta = (x_2,y_2,z_2,t_2)
  \end{equation*}
  in $R^4$ and any $c$ in $R$, we have
  \begin{align*}
    T(c\alpha + \beta)
    &= T(cx_1 + x_2, cy_1 + y_2, cz_1 + z_2, ct_1 + t_2) \\
    &=
    \begin{bmatrix}
      (ct_1 + t_2) + (cx_1 + x_2) & (cy_1 + y_2) + i(cz_1 + z_2) \\
      (cy_1 + y_2) - i(cz_1 + z_2) & (ct_1 + t_2) - (cx_1 + x_2)
    \end{bmatrix} \\
    &= c
    \begin{bmatrix}
      t_1 + x_1 & y_1 + iz_1 \\
      y_1 - iz_1 & t_1 - x_1
    \end{bmatrix}
    +
    \begin{bmatrix}
      t_2 + x_2 & y_2 + iz_2 \\
      y_2 - iz_2 & t_2 - x_2
    \end{bmatrix} \\
    &= cT\alpha + T\beta.
  \end{align*}
  This shows that $T$ is a linear transformation.

  Next, if $T\alpha = T\beta$ then $t_1 + x_1 = t_2 + x_2$ and
  $t_1 - x_1 = t_2 - x_2$, which together imply that $t_1 = t_2$ and
  $x_1 = x_2$. Similarly, $y_1 + iz_1 = y_2 + iz_2$ implies that
  $y_1 = y_2$ and $z_1 = z_2$. Therefore $T$ is one to one.

  Finally, let
  \begin{equation*}
    A =
    \begin{bmatrix}
      a & b + ci \\
      b - ci & d
    \end{bmatrix}
  \end{equation*}
  be any $2\times2$ Hermitian matrix. Then we see that
  \begin{equation*}
    T\left(\frac12a + \frac12d, b, c, \frac12a - \frac12d\right)
    =
    \begin{bmatrix}
      a & b + ci \\
      b - ci & d
    \end{bmatrix}
    = A,
  \end{equation*}
  so $T$ is onto. This shows that $T$ is an isomorphism and $R^4$ is
  isomorphic to $W$.
\end{proof}

\Exercise4 Show that $F^{m\times n}$ is isomorphic to $F^{mn}$.
\begin{proof}
  An obvious isomorphism is the map $T$ from $F^{m\times n}$ onto
  $F^{mn}$ given by
  \begin{multline*}
    T
    \begin{bmatrix}
      A_{11} & A_{12} & \cdots & A_{1n} \\
      A_{21} & A_{22} & \cdots & A_{2n} \\
      \vdots & \vdots & \ddots & \vdots \\
      A_{m1} & A_{m2} & \cdots & A_{mn}
    \end{bmatrix} \\
    =
    (A_{11},A_{12},\dots,A_{1n},A_{21},A_{22},\dots,A_{2n},\dots,
    A_{m1},A_{m2},\dots,A_{mn}).
  \end{multline*}
  That is, the $j$th coordinate of $T(A)$ is the $j$th entry of $A$
  when the entries are ordered from left-to-right and then
  top-to-bottom. It should be evident that $T$ is a linear
  transformation that is both one to one and onto.
\end{proof}

\Exercise5 Let $V$ be the set of complex numbers regarded as a vector
space over the field of real numbers. We define a function $T$ from
$V$ into the space of $2\times2$ real matrices, as follows. If
$z = x + iy$ with $x$ and $y$ real numbers, then
\begin{equation*}
  T(z) =
  \begin{bmatrix}
    x + 7y & 5y \\
    -10y & x - 7y
  \end{bmatrix}.
\end{equation*}
\begin{enumerate}
\item Verify that $T$ is a one-one (real) linear transformation of $V$
  into the space of $2\times2$ real matrices.
  \begin{proof}
    Let $z_1 = x_1 + iy_1$ and $z_2 = x_2 + iy_2$ where
    $x_1,x_2,y_1,y_2$ are real numbers. Let $c$ be any real
    number. Then
    \begin{align*}
      T(cz_1 + z_2)
      &= T(cx_1 + x_2 + i(cy_1 + y_2)) \\
      &=
      \begin{bmatrix}
        (cx_1 + x_2) + 7(cy_1 + y_2) & 5(cy_1 + y_2) \\
        -10(cy_1 + y_2) & (cx_1 + x_2) - 7(cy_1 + y_2)
      \end{bmatrix} \\
      &= c
      \begin{bmatrix}
        x_1 + 7y_1 & 5y_1 \\
        -10y_1 & x_1 - 7y_1
      \end{bmatrix}
      +
      \begin{bmatrix}
        x_2 + 7y_2 & 5y_2 \\
        -10y_2 & x_2 - 7y_2
      \end{bmatrix} \\
      &= cT(z_1) + T(z_2)
    \end{align*}
    and we see that $T$ is a linear transformation.

    Moreover, if $T(z_1) = T(z_2)$ then $5y_1 = 5y_2$ so that
    $y_1 = y_2$ and $x_1 = x_2$, or $z_1 = z_2$. Thus $T$ is one to
    one.
  \end{proof}

\item Verify that $T(z_1z_2) = T(z_1)T(z_2)$.
  \begin{proof}
    As above, we let $z_1 = x_1 + iy_1$ and $z_2 = x_2 + iy_2$. Then
    \begin{equation*}
      T(z_1)T(z_2)
      =
      \begin{bmatrix}
        x_1 + 7y_1 & 5y_1 \\
        -10y_1 & x_1 - 7y_1
      \end{bmatrix}
      \begin{bmatrix}
        x_2 + 7y_2 & 5y_2 \\
        -10y_2 & x_2 - 7y_2
      \end{bmatrix}.
    \end{equation*}
    If we calculate the upper-left entry of this matrix product, we
    find
    \begin{align*}
      [T(z_1)T(z_2)]_{11}
      &= (x_1 + 7y_1)(x_2 + 7y_2) - 50y_1y_2 \\
      &= x_1x_2 + 7x_1y_2 + 7x_2y_1 + 49y_1y_2 - 50y_1y_2 \\
      &= (x_1x_2 - y_1y_2) + 7(x_1y_2 + x_2y_1) \\
      &= [T(z_1z_2)]_{11}.
    \end{align*}
    The remaining entries are calculated in the same manner, and are
    all straightforward. Therefore $T(z_1z_2) = T(z_1)T(z_2)$.
  \end{proof}

\item How would you describe the range of $T$?
  \begin{solution}
    We have shown above that the range of $T$ is isomorphic to $V$. A
    basis for $V$ is $\{1, i\}$, so we may compute
    \begin{equation*}
      T(1) =
      \begin{bmatrix}
        1 & 0 \\
        0 & 1
      \end{bmatrix}
      \quad\text{and}\quad
      T(i) =
      \begin{bmatrix}
        7 & 5 \\
        -10 & -7
      \end{bmatrix}.
    \end{equation*}
    Since $T$ preserves linear independence (Theorem~8), we see that
    the set
    \begin{equation*}
      \left\{
        \begin{bmatrix}
          1 & 0 \\
          0 & 1
        \end{bmatrix},
        \begin{bmatrix}
          7 & 5 \\
          -10 & -7
        \end{bmatrix}
      \right\}
    \end{equation*}
    is a basis for the range of $T$.
  \end{solution}
\end{enumerate}

\Exercise6 Let $V$ and $W$ be finite-dimensional vector spaces over
the field $F$. Prove that $V$ and $W$ are isomorphic if and only if
$\dim V = \dim W$.
\begin{proof}
  First suppose that $V$ and $W$ are isomorphic via the isomorphism
  $T$. Suppose $V$ has dimension $n$ and let
  $\{\alpha_1,\dots,\alpha_n\}$ be a basis for $V$. Then by Theorem~8,
  the set $\{T\alpha_1,\dots,T\alpha_n\}$ is linearly independent and
  thus forms a basis for the range of $T$. But $T$ is onto, so the
  range of $T$ is $W$. Therefore $\dim W = n$ as required.

  Conversely, suppose $\dim V = \dim W = n$. By Theorem~10, $V$ and
  $W$ are both isomorphic to $F^n$. Thus $V$ is isomorphic to $W$ and
  the proof is complete.
\end{proof}

\Exercise7 Let $V$ and $W$ be vector spaces over the field $F$ and let
$U$ be an isomorphism of $V$ onto $W$. Prove that $T\to UTU^{-1}$ is
an isomorphism of $L(V, V)$ onto $L(W, W)$.
\begin{proof}
  Let $S$ denote the stated map from $L(V,V)$ to $L(W,W)$. If $c$ is
  in $F$, then
  \begin{align*}
    S(cT_1 + T_2)
    &= U(cT_1 + T_2)U^{-1} \\
    &= (cUT_1 + UT_2)U^{-1} \\
    &= cUT_1U^{-1} + UT_2U^{-1} \\
    &= cS(T_1) + S(T_2),
  \end{align*}
  so $S$ is a linear transformation.

  Next, suppose $S(T_1) = S(T_2)$. That is, $UT_1U^{-1} =
  UT_2U^{-1}$. Then
  \begin{align*}
    T_1 &= (U^{-1}U)T_1(U^{-1}U) \\
        &= U^{-1}(UT_1U^{-1})U \\
        &= U^{-1}(UT_2U^{-1})U \\
        &= (U^{-1}U)T_2(U^{-1}U) \\
        &= T_2,
  \end{align*}
  showing that $S$ is one to one.

  Finally, let $T$ be any linear operator in $L(W,W)$. Then
  \begin{equation*}
    S(U^{-1}TU) = U(U^{-1}TU)U^{-1} = (UU^{-1})T(UU^{-1}) = T,
  \end{equation*}
  so $S$ is onto. This shows that $S$ is an isomorphism, so that
  $L(V,V)$ is isomorphic to $L(W,W)$.
\end{proof}
