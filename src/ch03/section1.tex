\chapter{Linear Transformations}

\section{Linear Transformations}

\Exercise1 Which of the following functions $T$ from $R^2$ into $R^2$
are linear transformations?
\begin{enumerate}
\item $T(x_1, x_2) = (1 + x_1, x_2)$;
  \begin{solution}
    $T$ cannot be a linear transformation since
    \begin{equation*}
      T(0, 0) = (1, 0) \neq (0, 0). \qedhere
    \end{equation*}
  \end{solution}
\item $T(x_1, x_2) = (x_2, x_1)$;
  \begin{solution}
    $T$ is a linear transformation: let $c$ be a scalar and let
    $\alpha = (x_1, x_2)$ and $\beta = (y_1, y_2)$. Then
    \begin{align*}
      T(c\alpha + \beta)
      &= T(cx_1 + y_1, cx_2 + y_2) \\
      &= (cx_2 + y_2, cx_1 + y_1) \\
      &= c(x_2, x_1) + (y_2, y_1) \\
      &= cT(\alpha) + T(\beta). \qedhere
    \end{align*}
  \end{solution}
\item $T(x_1, x_2) = (x_1^2, x_2)$;
  \begin{solution}
    $T$ is not a linear transformation. For example,
    \begin{equation*}
      T(1, 0) + T(2, 0) = (1, 0) + (4, 0) = (5, 0)
    \end{equation*}
    but
    \begin{equation*}
      T((1, 0) + (2, 0)) = T(3, 0) = (9, 0). \qedhere
    \end{equation*}
  \end{solution}
\item $T(x_1, x_2) = (\sin x_1, x_2)$;
  \begin{solution}
    $T$ is not a linear transformation since
    \begin{equation*}
      T\left(\frac\pi2, 0\right) + T\left(\frac\pi2, 0\right)
      = (1, 0) + (1, 0) = (2, 0)
    \end{equation*}
    while
    \begin{equation*}
      T\left(\left(\frac\pi2, 0\right) + \left(\frac\pi2, 0\right)\right)
      = T(\pi, 0) = (0, 0). \qedhere
    \end{equation*}
  \end{solution}
\item $T(x_1, x_2) = (x_1 - x_2, 0)$.
  \begin{solution}
    Let $\alpha = (x_1, x_2)$ and $\beta = (y_1, y_2)$. Since
    \begin{align*}
      cT(\alpha) + T(\beta)
      &= c(x_1 - x_2, 0) + (y_1 - y_2, 0) \\
      &= (cx_1 - cx_2 + y_1 - y_2, 0) \\
      &= ((cx_1 + y_1) - (cx_2 + y_2), 0) \\
      &= T(cx_1 + y_1, cx_2 + y_2) \\
      &= T(c\alpha + \beta),
    \end{align*}
    we see that $T$ is a linear transformation.
  \end{solution}
\end{enumerate}

\Exercise2 Find the range, rank, null space, and nullity for the zero
transformation and the identity transformation on a finite-dimensional
space $V$.
\begin{solution}
  Let $V$ be a vector space of finite dimension $n$, let
  $T\colon V\to V$ be the zero transformation, and let
  $U\colon V\to V$ be the identity transformation.

  Then the range of $T$ is clearly the set consisting of the zero
  vector alone, and the null space is $V$ itself. The rank of $T$ is
  then $0$ (the zero subspace has the empty set as a basis) and the
  nullity of $T$ is $n$.

  For the identity transformation $U$, we see that the range is all of
  $V$, while the null space is the zero subspace $\{0\}$. Then the
  rank of $U$ is $n$ and the nullity of $U$ is $0$.

  Note that in each case, the rank plus the nullity is $n$, in
  agreement with Theorem~2.
\end{solution}

\Exercise3 Describe the range and the null space for the
differentiation transformation of Example~2. Do the same for the
integration transformation of Example~5.
\begin{solution}
  Let $V$ be the space of polynomial functions from $F$ into $F$ and
  let $D$ be the differentiation transformation. Given a polynomial
  \begin{equation*}
    f(x) = c_0 + c_1x + \cdots + c_kx^k,
  \end{equation*}
  we can always find another polynomial $g(x)$ such that
  $(Dg)(x) = f(x)$, namely the polynomial
  \begin{equation*}
    g(x) = c_0x + \frac12c_1x^2 + \cdots + \frac1{k+1}c_kx^{k+1}.
  \end{equation*}
  Therefore the range of $D$ is $V$.

  A function $f$ has zero derivative if and only if $f$ is constant,
  \begin{equation*}
    f(x) = c_0, \quad \text{for some $c_0$ in $F$.}
  \end{equation*}
  So the null space of $D$ is the space of constant functions.

  Now, let $T$ be the integration transformation defined in
  Example~5. We can integrate any polynomial to get another
  polynomial, but because of the manner in which $T$ was defined, the
  resulting polynomial will always have a constant term of $0$. Let
  $f$ be a polynomial with constant term zero,
  \begin{equation*}
    f(x) = c_1x + c_2x^2 + \cdots + c_kx^k.
  \end{equation*}
  Then the function $g(x) = (Df)(x)$ is such that $(Tg)(x) = f(x)$. So
  we see that the range of $T$ is the space of polynomials with zero
  constant term.

  Lastly, if $f$ is a polynomial such that $(Tf)(x) = 0$, then $f$
  must be the zero polynomial. The null space of $T$ is therefore the
  trivial subspace $\{0\}$.
\end{solution}

\Exercise4 Is there a linear transformation $T$ from $R^3$ into $R^2$
such that $T(1, -1, 1) = (1, 0)$ and $T(1, 1, 1) = (0, 1)$?
\begin{solution}
  Yes. In fact, there are infinitely many such transformations, as we
  will now show.

  Let $\alpha = (1, -1, 1)$ and let $\beta = (1, 1, 1)$. Since neither
  $\alpha$ nor $\beta$ is a multiple of the other, the set
  $\{\alpha, \beta\}$ is linearly independent. Therefore it can be
  extended to a basis for $R^3$. The existence of a linear
  transformation $T$ such that $T(\alpha) = (1, 0)$ and
  $T(\beta) = (0, 1)$ now follows from Theorem~1.

  To find such a transformation explicitly, we will form a basis for
  $R^3$. If we let $\gamma = (1, 0, 0)$, for example, it is not
  difficult to show that $\{\alpha, \beta, \gamma\}$ is a linearly
  independent set of vectors which spans $R^3$.

  We want to be able to write a vector $\rho = (b_1, b_2, b_3)$ as a
  linear combination of $\alpha$, $\beta$, and $\gamma$. To do this,
  we will set up an augmented matrix and perform row-reduction, as we
  have done before in Chapter~2:
  \begin{equation*}
    \begin{bmatrix}
      1 & 1 & 1 & y_1 \\
      -1 & 1 & 0 & y_2 \\
      1 & 1 & 0 & y_3
    \end{bmatrix}
    \rightarrow
    \begin{bmatrix}
      1 & 0 & 0 & \frac12y_3 - \frac12y_2 \\[3pt]
      0 & 1 & 0 & \frac12y_2 + \frac12y_3 \\[3pt]
      0 & 0 & 1 & y_1 - y_3
    \end{bmatrix}.
  \end{equation*}
  So we may write
  \begin{equation*}
    \rho = (b_1, b_2, b_3) =
    \frac12(b_3 - b_2)\alpha + \frac12(b_2 + b_3)\beta
    + (b_1 - b_3)\gamma.
  \end{equation*}

  Now suppose the transformation $T$ is such that
  $T(\gamma) = (x_1, x_2)$, for some $x_1$ and $x_2$ in $R$. Then we
  have
  \begin{align*}
    T(b_1, b_2, b_3)
    &= T\left(\frac12(b_3 - b_2)\alpha + \frac12(b_2 + b_3)\beta
      + (b_1 - b_3)\gamma\right) \\[3pt]
    &= \frac12(b_3 - b_2)T(\alpha) + \frac12(b_2 + b_3)T(\beta)
      + (b_1 - b_3)T(\gamma) \\[3pt]
    &= \frac12(b_3 - b_2, 0) + \frac12(0, b_2 + b_3)
      + (x_1b_1 - x_1b_3, x_2b_1 - x_2b_3) \\[3pt]
    &= \frac12(2x_1b_1 - b_2 + (1 - 2x_1)b_3,
      2x_2b_1 + b_2 + (1 - 2x_2)b_3).
  \end{align*}

  So, for example, taking
  $(x_1, x_2) = \left(\frac12, \frac12\right)$, we get
  \begin{equation*}
    T(b_1, b_2, b_3) = \left(\frac12b_1 - \frac12b_2,
      \frac12b_1 + \frac12b_2\right).
  \end{equation*}
  By picking different values for $x_1$ and $x_2$, we see that there
  are infinitely many possibilities for $T$.
\end{solution}

\Exercise5 If
\begin{align*}
  \alpha_1 &= (1, -1), \quad \beta_1 = (1, 0) \\
  \alpha_2 &= (2, -1), \quad \beta_2 = (0, 1) \\
  \alpha_3 &= (-3, 2), \quad \beta_3 = (1, 1)
\end{align*}
is there a linear transformation $T$ from $R^2$ into $R^2$ such that
$T\alpha_i = \beta_i$ for $i = 1$, $2$, and $3$?
\begin{solution}
  No. To see why, observe that
  \begin{equation*}
    (-3, 2) = -(1, -1) - (2, -1).
  \end{equation*}
  We have
  \begin{equation*}
    T(-\alpha_1 - \alpha_2) = (1, 1)
  \end{equation*}
  but
  \begin{equation*}
    -T(\alpha_1) - T(\alpha_2) = -(1, 0) - (0, 1) = (-1, -1).
  \end{equation*}
  So $T$ cannot be a linear transformation.
\end{solution}

\Exercise6 Describe explicitly the linear transformation $T$ from
$F^2$ into $F^2$ such that
\begin{equation*}
  T\epsilon_1 = (a, b), \quad T\epsilon_2 = (c, d).
\end{equation*}
\begin{solution}
  For any $(x_1, x_2)$ in $F^2$, we have
  \begin{align*}
    T(x_1, x_2)
    &= T(x_1\epsilon_1 + x_2\epsilon_2) \\
    &= x_1T(\epsilon_1) + x_2T(\epsilon_2) \\
    &= x_1(a, b) + x_2(c, d) \\
    &= (x_1a + x_2c, x_1b + x_2d). \qedhere
  \end{align*}
\end{solution}

\Exercise7 Let $F$ be a subfield of the complex numbers and let $T$ be
the function from $F^3$ into $F^3$ defined by
\begin{equation*}
  T(x_1, x_2, x_3) = (x_1 - x_2 + 2x_3, 2x_1 + x_2, -x_1 - 2x_2 + 2x_3).
\end{equation*}
\begin{enumerate}
\item Verify that $T$ is a linear transformation.
  \begin{solution}
    Let $\alpha = (x_1,x_2,x_3)$ and $\beta = (y_1,y_2,y_3)$. Also for
    $x$ in $F^3$, let $\pi_1(x)$ denote the first coordinate of $x$,
    $\pi_2(x)$ the second coordinate, and $\pi_3(x)$ the third
    coordinate. Then
    \begin{align*}
      \pi_1(cT(\alpha) + T(\beta))
      &= c(x_1 - x_2 + 2x_3) + (y_1 - y_2 + 2y_3) \\
      &= (cx_1 + y_1) - (cx_2 + y_2) + 2(cx_3 + y_3) \\
      &= \pi_1(T(c\alpha + \beta)), \\
      \pi_2(cT(\alpha) + T(\beta))
      &= c(2x_1 + x_2) + (2y_1 + y_2) \\
      &= 2(cx_1 + y_1) + (cx_2 + y_2) \\
      &= \pi_2(T(c\alpha + \beta)), \\
      \intertext{and}
      \pi_3(cT(\alpha) + T(\beta))
      &= c(-x_1 - 2x_2 + 2x_3) + (-y_1 - 2y_2 + 2y_3) \\
      &= -(cx_1 + y_1) - 2(cx_2 + y_2) + 2(cx_3 + y_3) \\
      &= \pi_3(T(c\alpha + \beta)).
    \end{align*}
    This shows that $T$ is a linear transformation.
  \end{solution}
\item If $(a, b, c)$ is a vector in $F^3$, what are the conditions on
  $a$, $b$, and $c$ that the vector be in the range of $T$? What is
  the rank of $T$?
  \begin{solution}
    If $(a, b, c)$ is in the range of $T$, then
    \begin{alignat*}{4}
      x_1 &{}-{}& x_2 &{}+{}& 2x_3 &{}={}& a \\
      2x_1 &{}+{}& x_2 && &{}={}& b \\
      -x_1 &{}-{}& 2x_2 &{}+{}& 2x_3 &{}={}& c\rlap.
    \end{alignat*}
    In performing row-reduction on the augmented matrix for the above
    system, we get
    \begin{equation*}
      \begin{bmatrix}
        1 & -1 & 2 & a \\
        2 & 1 & 0 & b\\
        -1 & -2 & 2 & c
      \end{bmatrix}
      \rightarrow
      \begin{bmatrix}
        1 & 0 & \frac23 & \frac13a + \frac13b \\[3pt]
        0 & 1 & -\frac43 & -\frac23a + \frac13b \\[3pt]
        0 & 0 & 0 & -a + b + c
      \end{bmatrix}.
    \end{equation*}
    From this latter matrix, we see that this system of equations has
    a solution if and only if
    \begin{equation*}
      -a + b + c = 0.
    \end{equation*}

    We also see that the coefficient matrix
    \begin{equation*}
      A =
      \begin{bmatrix}
        1 & -1 & 2 \\
        2 & 1 & 0 \\
        -1 & -2 & 2
      \end{bmatrix}
    \end{equation*}
    has a row rank (and thus column rank) of $2$. But the column space
    of $A$ is precisely the range of $T$, so we may conclude that $T$
    has rank $2$.
  \end{solution}
\item What are the conditions on $a$, $b$, and $c$ that $(a,b,c)$ be
  in the null space of $T$? What is the nullity of $T$?
  \begin{solution}
    $(a, b, c)$ is in the null space of $T$ if and only if
    \begin{alignat*}{4}
      a &{}-{}& b &{}+{}& 2c &{}={}& 0 \\
      2a &{}+{}& b && &{}={}& 0 \\
      -a &{}-{}& 2b &{}+{}& 2c &{}={}& 0\rlap.
    \end{alignat*}
    We have already seen above that the coefficient matrix reduces to
    \begin{equation*}
      \begin{bmatrix}
        1 & -1 & 2 \\
        2 & 1 & 0\\
        -1 & -2 & 2
      \end{bmatrix}
      \rightarrow
      \begin{bmatrix}
        1 & 0 & \frac23 \\[3pt]
        0 & 1 & -\frac43 \\[3pt]
        0 & 0 & 0
      \end{bmatrix}.
    \end{equation*}
    So $(a, b, c)$ is in the null space if and only if $a + 2c/3 = 0$
    and $b - 4c/3 = 0$.

    Letting $c = -3$, for example, we find one possible basis for the
    null space of $T$ to be $\{(2, -4, -3)\}$. We see that the nullity
    of $T$ is therefore $1$, which is as we should expect since $F^3$
    has dimension $3$ and the rank of $T$ is $2$.
  \end{solution}
\end{enumerate}

\Exercise8 Describe explicitly a linear transformation from $R^3$ into
$R^3$ which has as its range the subspace spanned by $(1, 0, -1)$ and
$(1, 2, 2)$.
\begin{solution}
  Let $\{\epsilon_1, \epsilon_2, \epsilon_3\}$ denote the standard
  ordered basis for $R^3$. Theorem~1 allows us to find infinitely many
  linear transformations satisfying the given criterion. For example,
  we may take some linear combination of the two given vectors, say
  $(2, 2, 1)$, and then look for a linear transformation $T$ such
  that
  \begin{equation*}
    T\epsilon_1 = (1, 0, -1),
    \quad
    T\epsilon_2 = (1, 2, 2),
    \quad\text{and}\quad
    T\epsilon_3 = (2, 2, 1).
  \end{equation*}
  Evidently, the transformation
  \begin{equation*}
    T(x_1, x_2, x_3) =
    (x_1 + x_2 + 2x_3, 2x_2 + 2x_3, -x_1 + 2x_2 + x_3)
  \end{equation*}
  does the job. The range of $T$ is precisely the subspace of $R^3$
  spanned by $(1, 0, -1)$ and $(1, 2, 2)$. Of course, as noted, there
  are infinitely many other transformations that would work.
\end{solution}

\Exercise9 Let $V$ be the vector space of all $n\times n$ matrices
over the field $F$, and let $B$ be a fixed $n\times n$ matrix. If
\begin{equation*}
  T(A) = AB - BA
\end{equation*}
verify that $T$ is a linear transformation from $V$ into $V$.
\begin{proof}
  Let $A_1$ and $A_2$ be members of $V$ and let $c$ be a scalar in
  $F$. Using the ordinary properties of matrix addition and
  multiplication, we have
  \begin{align*}
    cT(A_1) + T(A_2)
    &= c(A_1B - BA_1) + (A_2B - BA_2) \\
    &= (cA_1 + A_2)B - B(cA_1 + A_2) \\
    &= T(cA_1 + A_2).
  \end{align*}
  Therefore $T$ is a linear transformation from $V$ into $V$.
\end{proof}

\Exercise{10} Let $V$ be the set of all complex numbers regarded as a
vector space over the field of real numbers (usual operations). Find a
function from $V$ into $V$ which is a linear transformation on the
above vector space, but which is not a linear transformation on $C^1$,
i.e., which is not complex linear.
\begin{solution}
  Define $T(a + bi) = a$. Then for any real number $c$ and any
  complex numbers $z = a_1 + b_1i$ and $w = a_2 + b_2i$, we have
  \begin{equation*}
    cT(z) + T(w) = ca_1 + a_2 = T(cz + w),
  \end{equation*}
  so $T$ is a linear transformation from $V$ into $V$. However,
  \begin{equation*}
    iT(1) = i \neq 1 = T(i),
  \end{equation*}
  so $T$ is {\em not} linear on $C^1$.
\end{solution}
