\chapter{Linear Transformations}

\section{Linear Transformations}

\Exercise1 Which of the following functions $T$ from $R^2$ into $R^2$
are linear transformations?
\begin{enumerate}
\item $T(x_1, x_2) = (1 + x_1, x_2)$;
  \begin{solution}
    $T$ cannot be a linear transformation since
    \begin{equation*}
      T(0, 0) = (1, 0) \neq (0, 0). \qedhere
    \end{equation*}
  \end{solution}
\item $T(x_1, x_2) = (x_2, x_1)$;
  \begin{solution}
    $T$ is a linear transformation: let $c$ be a scalar and let
    $\alpha = (x_1, x_2)$ and $\beta = (y_1, y_2)$. Then
    \begin{align*}
      T(c\alpha + \beta)
      &= T(cx_1 + y_1, cx_2 + y_2) \\
      &= (cx_2 + y_2, cx_1 + y_1) \\
      &= c(x_2, x_1) + (y_2, y_1) \\
      &= cT(\alpha) + T(\beta). \qedhere
    \end{align*}
  \end{solution}
\item $T(x_1, x_2) = (x_1^2, x_2)$;
  \begin{solution}
    $T$ is not a linear transformation. For example,
    \begin{equation*}
      T(1, 0) + T(2, 0) = (1, 0) + (4, 0) = (5, 0)
    \end{equation*}
    but
    \begin{equation*}
      T((1, 0) + (2, 0)) = T(3, 0) = (9, 0). \qedhere
    \end{equation*}
  \end{solution}
\item $T(x_1, x_2) = (\sin x_1, x_2)$;
  \begin{solution}
    $T$ is not a linear transformation since
    \begin{equation*}
      T\left(\frac\pi2, 0\right) + T\left(\frac\pi2, 0\right)
      = (1, 0) + (1, 0) = (2, 0)
    \end{equation*}
    while
    \begin{equation*}
      T\left(\left(\frac\pi2, 0\right) + \left(\frac\pi2, 0\right)\right)
      = T(\pi, 0) = (0, 0). \qedhere
    \end{equation*}
  \end{solution}
\item $T(x_1, x_2) = (x_1 - x_2, 0)$.
  \begin{solution}
    Let $\alpha = (x_1, x_2)$ and $\beta = (y_1, y_2)$. Since
    \begin{align*}
      cT(\alpha) + T(\beta)
      &= c(x_1 - x_2, 0) + (y_1 - y_2, 0) \\
      &= (cx_1 - cx_2 + y_1 - y_2, 0) \\
      &= ((cx_1 + y_1) - (cx_2 + y_2), 0) \\
      &= T(cx_1 + y_1, cx_2 + y_2) \\
      &= T(c\alpha + \beta),
    \end{align*}
    we see that $T$ is a linear transformation.
  \end{solution}
\end{enumerate}

\Exercise2 Find the range, rank, null space, and nullity for the zero
transformation and the identity transformation on a finite-dimensional
space $V$.
\begin{solution}
  Let $V$ be a vector space of finite dimension $n$, let
  $T\colon V\to V$ be the zero transformation, and let
  $U\colon V\to V$ be the identity transformation.

  Then the range of $T$ is clearly the set consisting of the zero
  vector alone, and the null space is $V$ itself. The rank of $T$ is
  then $0$ (the zero subspace has the empty set as a basis) and the
  nullity of $T$ is $n$.

  For the identity transformation $U$, we see that the range is all of
  $V$, while the null space is the zero subspace $\{0\}$. Then the
  rank of $U$ is $n$ and the nullity of $U$ is $0$.

  Note that in each case, the rank plus the nullity is $n$, in
  agreement with Theorem~2.
\end{solution}

\Exercise3 Describe the range and the null space for the
differentiation transformation of Example~2. Do the same for the
integration transformation of Example~5.
\begin{solution}
  Let $V$ be the space of polynomial functions from $F$ into $F$ and
  let $D$ be the differentiation transformation. Given a polynomial
  \begin{equation*}
    f(x) = c_0 + c_1x + \cdots + c_kx^k,
  \end{equation*}
  we can always find another polynomial $g(x)$ such that
  $(Dg)(x) = f(x)$, namely the polynomial
  \begin{equation*}
    g(x) = c_0x + \frac12c_1x^2 + \cdots + \frac1{k+1}c_kx^{k+1}.
  \end{equation*}
  Therefore the range of $D$ is $V$.

  A function $f$ has zero derivative if and only if $f$ is constant,
  \begin{equation*}
    f(x) = c_0, \quad \text{for some $c_0$ in $F$.}
  \end{equation*}
  So the null space of $D$ is the space of constant functions.

  Now, let $T$ be the integration transformation defined in
  Example~5. We can integrate any polynomial to get another
  polynomial, but because of the manner in which $T$ was defined, the
  resulting polynomial will always have a constant term of $0$. Let
  $f$ be a polynomial with constant term zero,
  \begin{equation*}
    f(x) = c_1x + c_2x^2 + \cdots + c_kx^k.
  \end{equation*}
  Then the function $g(x) = (Df)(x)$ is such that $(Tg)(x) = f(x)$. So
  we see that the range of $T$ is the space of polynomials with zero
  constant term.

  Lastly, if $f$ is a polynomial such that $(Tf)(x) = 0$, then $f$
  must be the zero polynomial. The null space of $T$ is therefore the
  trivial subspace $\{0\}$.
\end{solution}
