\section{The Algebra of Linear Transformations}

\Exercise1 Let $T$ and $U$ be the linear operators on $R^2$ defined by
\begin{equation*}
  T(x_1,x_2) = (x_2,x_1) \quad\text{and}\quad
  U(x_1,x_2) = (x_1,0).
\end{equation*}
\begin{enumerate}
\item How would you describe $T$ and $U$ geometrically?
  \begin{solution}
    $T$ mirrors points across the line $y = x$. $U$ projects points
    onto the $x$-axis.
  \end{solution}

\item Give rules like the ones defining $T$ and $U$ for each of the
  transformations $(U + T)$, $UT$, $TU$, $T^2$, $U^2$.
  \begin{solution}
    We have
    \begin{align*}
      (U + T)(x_1,x_2) &= (x_1 + x_2, x_1), \\
      UT(x_1,x_2) &= (x_2, 0), \\
      TU(x_1,x_2) &= (0, x_1), \\
      T^2(x_1,x_2) &= (x_1,x_2), \\
      \intertext{and}
      U^2(x_1,x_2) &= (x_1,0). \qedhere
    \end{align*}
  \end{solution}
\end{enumerate}

\Exercise2 Let $T$ be the (unique) linear operator on $C^3$ for which
\begin{equation*}
  T\epsilon_1 = (1, 0, i), \quad
  T\epsilon_2 = (0, 1, 1), \quad
  T\epsilon_3 = (i, 1, 0).
\end{equation*}
Is $T$ invertible?
\begin{solution}
  Let $\alpha = (z_1,z_2,z_3)$ be a vector in $C^3$ such that
  $T\alpha = 0$. Then
  \begin{equation*}
    z_1(1, 0, i) + z_2(0, 1, 1) + z_3(i, 1, 0) = (0, 0, 0),
  \end{equation*}
  or
  \begin{alignat*}{4}
    z_1 && &{}+{}& z_3i &{}={}& 0 \\
    && z_2 &{}+{}& z_3 &{}={}& 0 \\
    z_1i &{}+{}& z_2 && &{}={}& 0\rlap.
  \end{alignat*}
  This system of equations has infinitely many solutions, each of the
  form
  \begin{equation*}
    (z_1,z_2,z_3) = (-ti, -t, t).
  \end{equation*}
  Therefore the null space of $T$ is not $\{0\}$, so $T$ is singular
  and {\em not} invertible.
\end{solution}

\Exercise3
\label{exercise:lin-tran:inv-lin-op-R3-ex}
Let $T$ be the linear operator on $R^3$ defined by
\begin{equation*}
  T(x_1,x_2,x_3) = (3x_1, x_1 - x_2, 2x_1 + x_2 + x_3).
\end{equation*}
Is $T$ invertible? If so, find a rule for $T^{-1}$ like the one which
defines $T$.
\begin{solution}
  It is not difficult to see that $T$ is non-singular since
  \begin{equation*}
    T(x_1,x_2,x_3) = (0,0,0)
    \quad\text{if and only if}\quad
    x_1 = x_2 = x_3 = 0.
  \end{equation*}
  By Theorem~9, $T$ is invertible.

  Let $\alpha = (y_1,y_2,y_3)$ in $R^3$ be such that
  $T\alpha = (x_1,x_2,x_3)$. Then
  \begin{alignat*}{4}
    3y_1 && && &{}={}& x_1 \\
    y_1 &{}-{}& y_2 && &{}={}& x_2 \\
    2y_1 &{}+{}& y_2 &{}+{}& y_3 &{}={}& x_3\rlap.
  \end{alignat*}
  We see that this system of equations has the unique solution
  \begin{equation*}
    (y_1,y_2,y_3)
    = \left(\frac13x_1, \frac13x_1 - x_2, x_3 - x_1 + x_2\right).
  \end{equation*}
  So we have
  \begin{equation*}
    T^{-1}(x_1,x_2,x_3)
    = \left(\frac13x_1, \frac13x_1 - x_2, x_3 - x_1 + x_2\right).
    \qedhere
  \end{equation*}
\end{solution}

\Exercise4 For the linear operator $T$ of
Exercise~\ref{exercise:lin-tran:inv-lin-op-R3-ex}, prove that
\begin{equation*}
  (T^2 - I)(T - 3I) = 0.
\end{equation*}
\begin{proof}
  We have
  \begin{align*}
    T^2(x_1,x_2,x_3)
    &= T(3x_1, x_1 - x_2, 2x_1 + x_2 + x_3) \\
    &= (9x_1, 2x_1 + x_2, 9x_1 + x_3),
  \end{align*}
  so
  \begin{equation*}
    (T^2 - I)(x_1,x_2,x_3)
    = (8x_1, x_1, 8x_1).
  \end{equation*}
  Also,
  \begin{equation*}
    (T - 3I)(x_1, x_2, x_3)
    = (0, x_1 - 4x_2, 2x_1 + x_2 - 2x_3).
  \end{equation*}
  Consequently,
  \begin{align*}
    (T^2 - I)(T - 3I)(x_1,x_2,x_3)
    &= (T^2 - I)(0, x_1 - 4x_2, 2x_1 + x_2 - 2x_3) \\
    &= (0, 0, 0).
  \end{align*}
  Therefore $(T^2 - I)(T - 3I) = 0$.
\end{proof}
