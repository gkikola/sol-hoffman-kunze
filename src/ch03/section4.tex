\section{Representation of Transformations by Matrices}

\Exercise1 Let $T$ be the linear operator on $C^2$ defined by
$T(x_1, x_2) = (x_1, 0)$. Let $\mathcal{B}$ be the standard ordered
basis for $C^2$ and let $\mathcal{B}' = \{\alpha_1,\alpha_2\}$ be the
ordered basis defined by $\alpha_1 = (1, i)$, $\alpha_2 = (-i, 2)$.
\begin{enumerate}
\item What is the matrix of $T$ relative to the pair $\mathcal{B}$,
  $\mathcal{B}'$?
  \begin{solution}
    We have
    \begin{equation*}
      T(1, 0) = (1, 0)
      \quad\text{and}\quad
      T(0, 1) = (0, 0).
    \end{equation*}
    Now let
    \begin{equation*}
      P =
      \begin{bmatrix}
        1 & -i \\
        i & 2
      \end{bmatrix}.
    \end{equation*}
    Then
    \begin{equation*}
      P^{-1} =
      \begin{bmatrix}
        2 & i \\
        -i & 1
      \end{bmatrix}
    \end{equation*}
    and we get
    \begin{equation*}
      [(1, 0)]_{\mathcal{B}'}
      = P^{-1}[(1,0)]_{\mathcal{B}}
      =
      \begin{bmatrix}
        2 & i \\
        -i & 1
      \end{bmatrix}
      \begin{bmatrix}
        1 \\ 0
      \end{bmatrix}
      =
      \begin{bmatrix}
        2 \\ -i
      \end{bmatrix}.
    \end{equation*}
    Of course, the zero vector has the same coordinates in every
    basis, so we see that the matrix of $T$ relative to
    $\mathcal{B},\mathcal{B}'$ is
    \begin{equation*}
      [T]_{\mathcal{B}}^{\mathcal{B}'} =
      \begin{bmatrix}
        2 & 0 \\
        -i & 0
      \end{bmatrix}. \qedhere
    \end{equation*}
  \end{solution}

\item What is the matrix of $T$ relative to the pair $\mathcal{B}'$,
  $\mathcal{B}$?
  \begin{solution}
    We have
    \begin{equation*}
      T(1,i) = (1,0) \quad\text{and}\quad
      T(-i, 2) = (-i,0).
    \end{equation*}
    So the matrix of $T$ relative to $\mathcal{B}',\mathcal{B}$ is
    \begin{equation*}
      [T]_{\mathcal{B}'}^{\mathcal{B}} =
      \begin{bmatrix}
        1 & -i \\
        0 & 0
      \end{bmatrix}. \qedhere
    \end{equation*}
  \end{solution}

\item What is the matrix of $T$ in the ordered basis $\mathcal{B}'$?
  \begin{solution}
    We have
    \begin{equation*}
      [T]_{\mathcal{B}'} = P^{-1}[T]_{\mathcal{B}}P =
      \begin{bmatrix}
        2 & i \\
        -i & 1
      \end{bmatrix}
      \begin{bmatrix}
        1 & 0 \\
        0 & 0
      \end{bmatrix}
      \begin{bmatrix}
        1 & -i \\
        i & 2
      \end{bmatrix} =
      \begin{bmatrix}
        2 & -2i \\
        -i & -1
      \end{bmatrix}. \qedhere
    \end{equation*}
  \end{solution}

\item What is the matrix of $T$ in the ordered basis
  $\{\alpha_2, \alpha_1\}$?
  \begin{solution}
    The change-of-basis matrix $P$ such that
    \begin{equation*}
      P[\alpha]_{\{\alpha_2,\alpha_1\}} = [\alpha]_{\mathcal{B}'}
    \end{equation*}
    is given by
    \begin{equation*}
      P =
      \begin{bmatrix}
        0 & 1 \\
        1 & 0
      \end{bmatrix},
    \end{equation*}
    so
    \begin{equation*}
      [T]_{\{\alpha_2,\alpha_1\}}
      = P^{-1}[T]_{\mathcal{B}'}P
      =
      \begin{bmatrix}
        0 & 1 \\
        1 & 0
      \end{bmatrix}
      \begin{bmatrix}
        2 & -2i \\
        -i & -1
      \end{bmatrix}
      \begin{bmatrix}
        0 & 1 \\
        1 & 0
      \end{bmatrix}
      =
      \begin{bmatrix}
        -1 & -i \\
        -2i & 2
      \end{bmatrix}. \qedhere
    \end{equation*}
  \end{solution}
\end{enumerate}

\Exercise2 Let $T$ be the linear transformation from $R^3$ into $R^2$
defined by
\begin{equation*}
  T(x_1,x_2,x_3) = (x_1 + x_2, 2x_3 - x_1).
\end{equation*}
\begin{enumerate}
\item If $\mathcal{B}$ is the standard ordered basis for $R^3$ and
  $\mathcal{B}'$ is the standard ordered basis for $R^2$, what is the
  matrix of $T$ relative to the pair $\mathcal{B},\mathcal{B}'$?
  \begin{solution}
    Since
    \begin{equation*}
      T(1, 0, 0) = (1, -1), \quad
      T(0, 1, 0) = (1, 0), \quad\text{and}\quad
      T(0, 0, 1) = (0, 2),
    \end{equation*}
    we see that the matrix of $T$ relative to
    $\mathcal{B},\mathcal{B}'$ is
    \begin{equation*}
      [T]_{\mathcal{B}}^{\mathcal{B}'} =
      \begin{bmatrix}
        1 & 1 & 0 \\
        -1 & 0 & 2
      \end{bmatrix}. \qedhere
    \end{equation*}
  \end{solution}

\item If $\mathcal{B} = \{\alpha_1,\alpha_2,\alpha_3\}$ and
  $\mathcal{B}' = \{\beta_1,\beta_2\}$, where
  \begin{multline*}
    \alpha_1 = (1, 0, -1), \quad
    \alpha_2 = (1, 1, 1), \quad
    \alpha_3 = (1, 0, 0), \\
    \beta_1 = (0, 1), \quad
    \beta_2 = (1, 0)
  \end{multline*}
  what is the matrix of $T$ relative to the pair
  $\mathcal{B},\mathcal{B}'$?
  \begin{solution}
    We have
    \begin{align*}
      T\alpha_1 &= (1, -3) = -3\beta_1 + \beta_2, \\
      T\alpha_2 &= (2, 1) = \beta_1 + 2\beta_2, \\
      \intertext{and}
      T\alpha_3 &= (1, -1) = -\beta_1 + \beta_2,
    \end{align*}
    so the corresponding matrix is
    \begin{equation*}
      [T]_{\mathcal{B}}^{\mathcal{B}'} =
      \begin{bmatrix}
        -3 & 1 & -1 \\
        1 & 2 & 1
      \end{bmatrix}. \qedhere
    \end{equation*}
  \end{solution}
\end{enumerate}

\Exercise3 Let $T$ be a linear operator on $F^n$, let $A$ be the
matrix of $T$ in the standard ordered basis for $F^n$, and let $W$ be
the subspace of $F^n$ spanned by the column vectors of $A$. What does
$W$ have to do with $T$?
\begin{solution}
  $W$ is simply the range of $T$, as we will now show.

  Let $\{\epsilon_1,\dots,\epsilon_n\}$ denote the standard ordered
  basis for $F^n$. Note that the $j$th column of $A$ is simply
  $T\epsilon_j$. Take any vector $\alpha$ in $F^n$. Then $\alpha$
  belongs to $W$ if and only if
  \begin{align*}
    \alpha
    &= x_1T\epsilon_1 + x_2T\epsilon_2 + \cdots + x_nT\epsilon_n \\
    &= T(x_1\epsilon_1 + x_2\epsilon_2 + \cdots + x_n\epsilon_n) \\
    &= T(x_1,x_2,\cdots,x_n),
  \end{align*}
  for some vector $(x_1,x_2,\dots,x_n)$ in $F^n$. That is, $\alpha$ is
  in $W$ if and only if $\alpha$ is in the range of $T$.
\end{solution}

\Exercise4 Let $V$ be a two-dimensional vector space over the field
$F$, and let $\mathcal{B}$ be an ordered basis for $V$. If $T$ is a
linear operator on $V$ and
\begin{equation*}
  [T]_{\mathcal{B}} =
  \begin{bmatrix}
    a & b \\
    c & d
  \end{bmatrix}
\end{equation*}
prove that $T^2 - (a + d)T + (ad - bc)I = 0$.
\begin{proof}
  By Theorem~12, the function which assigns a linear operator on $V$
  to its matrix relative to $\mathcal{B}$ is an isomorphism between
  $L(V, V)$ and $F^{2\times 2}$. Theorem~13 shows that this function
  preserves products also. Thus we can operate on $T$ by simply
  performing the corresponding operations on its matrix and vice
  versa. So consider the following computation.
  \begin{multline*}
    [T]_{\mathcal{B}}^2 - (a + d)[T]_{\mathcal{B}} + (ad - bc)I \\
    \begin{aligned}
      &=
      \begin{bmatrix}
        a^2 + bc & ab + bd \\
        ac + cd & bc + d^2
      \end{bmatrix}
      -
      \begin{bmatrix}
        a^2 + ad & ab + bd \\
        ac + cd & ad + d^2
      \end{bmatrix}
      +
      \begin{bmatrix}
        ad - bc & 0 \\
        0 & ad - bc
      \end{bmatrix} \\
      &=
      \begin{bmatrix}
        0 & 0 \\
        0 & 0
      \end{bmatrix}.
    \end{aligned}
  \end{multline*}
  From this we see that $T^2 - (a + d)T + (ad - bc)I = 0$.
\end{proof}
