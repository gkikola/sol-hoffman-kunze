\section{The Transpose of a Linear Transformation}

\Exercise1 Let $F$ be a field and let $f$ be the linear functional on
$F^2$ defined by
\begin{equation*}
  f(x_1,x_2) = ax_1 + bx_2.
\end{equation*}
For each of the following linear operators $T$, let $g = T^tf$, and
find $g(x_1,x_2)$.
\begin{enumerate}
\item $T(x_1,x_2) = (x_1,0)$
  \begin{solution}
    We have
    \begin{align*}
      g(x_1,x_2)
      &= (T^tf)(x_1,x_2) \\
      &= f(T(x_1,x_2)) \\
      &= f(x_1,0) \\
      &= ax_1. \qedhere
    \end{align*}
  \end{solution}

\item $T(x_1,x_2) = (-x_2,x_1)$
  \begin{solution}
    In this case, we get
    \begin{align*}
      g(x_1,x_2)
      &= f(T(x_1,x_2)) \\
      &= f(-x_2,x_1) \\
      &= -ax_2 + bx_1. \qedhere
    \end{align*}
  \end{solution}

\item $T(x_1,x_2) = (x_1 - x_2, x_1 + x_2)$
  \begin{solution}
    \begin{align*}
      g(x_1,x_2)
      &= f(x_1 - x_2, x_1 + x_2) \\
      &= (b + a)x_1 + (b - a)x_2. \qedhere
    \end{align*}
  \end{solution}
\end{enumerate}

\Exercise2 Let $V$ be the vector space of all polynomial functions
over the field of real numbers. Let $a$ and $b$ be fixed real numbers
and let $f$ be the linear functional on $V$ defined by
\begin{equation*}
  f(p) = \int_a^bp(x)\,dx.
\end{equation*}
If $D$ is the differentiation operator on $V$, what is $D^tf$?
\begin{solution}
  From the definition, we have
  \begin{equation*}
    (D^tf)(p)
    = f(Dp) \\
    = \int_a^b(Dp)(x)\,dx.
  \end{equation*}
  So, the fundamental theorem of calculus gives
  \begin{equation*}
    (D^tf)(p) = p(b) - p(a). \qedhere
  \end{equation*}
\end{solution}

\Exercise3 Let $V$ be the space of all $n\times n$ matrices over a
field $F$ and let $B$ be a fixed $n\times n$ matrix. If $T$ is the
linear operator on $V$ defined by $T(A) = AB - BA$, and if $f$ is the
trace function, what is $T^tf$?
\begin{solution}
  From Exercise~\ref{exercise:lin-tran:trace-AB-eq-trace-BA}, we know
  that $f(AB) = f(BA)$. So,
  \begin{align*}
    (T^tf)(A)
    &= f(TA) \\
    &= f(AB - BA) \\
    &= f(AB) - f(BA) \\
    &= 0,
  \end{align*}
  and we see that $T^tf = 0$.
\end{solution}

\Exercise4 Let $V$ be a finite-dimensional vector space over the field
$F$ and let $T$ be a linear operator on $V$. Let $c$ be a scalar and
suppose there is a non-zero vector $\alpha$ in $V$ such that
$T\alpha = c\alpha$. Prove that there is a non-zero linear functional
$f$ on $V$ such that $T^tf = cf$.
\begin{proof}
  Let $n = \dim V$ and let $\alpha$ be a nonzero vector in $V$ with
  $T\alpha = c\alpha$. Define the linear operator $U$ on $V$ by
  \begin{equation*}
    U = T - cI.
  \end{equation*}
  Then $U\alpha = T\alpha - c\alpha = 0$. Therefore $\alpha$ belongs
  to the null space of $U$, which implies (by Theorem~2) that
  $\rank(U) < n$. Now consider the linear operator $U^t$ on $V^*$. By
  Theorem~22, we have
  \begin{equation*}
    \rank(U^t) = \rank(U) < n,
  \end{equation*}
  so the nullspace of $U^t$ has dimension greater than zero. Therefore
  we can find a nonzero linear functional $f$ in $V^*$ such that
  $U^tf = 0$. Then for any $\beta$ in $V$,
  \begin{align*}
    0
    &= (U^tf)(\beta) \\
    &= f(U\beta) \\
    &= f(T\beta - c\beta) \\
    &= f(T\beta) - cf(\beta) \\
    &= (T^tf)(\beta) - cf(\beta).
  \end{align*}
  So, we have $T^tf = cf$ as required.
\end{proof}

\Exercise5 Let $A$ be an $m\times n$ matrix with {\em real}
entries. Prove that $A = 0$ if and only if $\trace(A^tA) = 0$.
\begin{proof}
  Certainly if $A = 0$, then $A^tA = 0$ and $\trace(A^tA) = 0$. We now
  need only prove the converse. Let $B = A^t$ and suppose
  $\trace(BA) = 0$. Then
  \begin{equation*}
    0
    = \sum_{k=1}^n(BA)_{kk}
    = \sum_{k=1}^n\sum_{\ell=1}^mB_{k\ell}A_{\ell k}
    = \sum_{k=1}^n\sum_{\ell=1}^mA_{\ell k}^2.
  \end{equation*}
  Since we have a sum of squares (of real numbers) equal to zero, it
  must be the case that each squared number is zero. In particular
  $A_{ij} = 0$ for each $i,j$, since every such entry appears in the
  sum.
\end{proof}

\Exercise6 Let $n$ be a positive integer and let $V$ be the space of
all polynomial functions over the field of real numbers which have
degree at most $n$, i.e., functions of the form
\begin{equation*}
  f(x) = c_0 + c_1x + \cdots + c_nx^n.
\end{equation*}
Let $D$ be the differentiation operator on $V$. Find a basis for the
null space of the transpose operator $D^t$.
\begin{solution}
  Let $W$ denote the range of $D$. By Theorem~22, the null space of
  $D^t$ is the annihilator of $W$. We know that $W$ is the space of
  polynomials having degree at most $n - 1$, so $\dim W = n - 1$. By
  Theorem~16, the annihilator $W^0$ must have dimension
  $n - (n - 1) = 1$, so we may take any nonzero linear functional in
  $W^0$ as a basis vector. Let $g$ be the unique linear functional
  such that
  \begin{equation*}
    g(x^k) = \delta_{nk}, \quad 1\leq k\leq n.
  \end{equation*}
  That is, $g$ sends every polynomial to the coefficient for its $x^n$
  term. In particular, $g$ annihilates $W$, so $\{g\}$ is a basis for
  $W^0$ and therefore also a basis for the null space of $D^t$.
\end{solution}
