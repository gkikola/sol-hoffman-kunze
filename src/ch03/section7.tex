\section{The Transpose of a Linear Transformation}

\Exercise1 Let $F$ be a field and let $f$ be the linear functional on
$F^2$ defined by
\begin{equation*}
  f(x_1,x_2) = ax_1 + bx_2.
\end{equation*}
For each of the following linear operators $T$, let $g = T^tf$, and
find $g(x_1,x_2)$.
\begin{enumerate}
\item $T(x_1,x_2) = (x_1,0)$
  \begin{solution}
    We have
    \begin{align*}
      g(x_1,x_2)
      &= (T^tf)(x_1,x_2) \\
      &= f(T(x_1,x_2)) \\
      &= f(x_1,0) \\
      &= ax_1. \qedhere
    \end{align*}
  \end{solution}

\item $T(x_1,x_2) = (-x_2,x_1)$
  \begin{solution}
    In this case, we get
    \begin{align*}
      g(x_1,x_2)
      &= f(T(x_1,x_2)) \\
      &= f(-x_2,x_1) \\
      &= -ax_2 + bx_1. \qedhere
    \end{align*}
  \end{solution}

\item $T(x_1,x_2) = (x_1 - x_2, x_1 + x_2)$
  \begin{solution}
    \begin{align*}
      g(x_1,x_2)
      &= f(x_1 - x_2, x_1 + x_2) \\
      &= (b + a)x_1 + (b - a)x_2. \qedhere
    \end{align*}
  \end{solution}
\end{enumerate}

\Exercise2 Let $V$ be the vector space of all polynomial functions
over the field of real numbers. Let $a$ and $b$ be fixed real numbers
and let $f$ be the linear functional on $V$ defined by
\begin{equation*}
  f(p) = \int_a^bp(x)\,dx.
\end{equation*}
If $D$ is the differentiation operator on $V$, what is $D^tf$?
\begin{solution}
  From the definition, we have
  \begin{equation*}
    (D^tf)(p)
    = f(Dp) \\
    = \int_a^b(Dp)(x)\,dx.
  \end{equation*}
  So, the fundamental theorem of calculus gives
  \begin{equation*}
    (D^tf)(p) = p(b) - p(a). \qedhere
  \end{equation*}
\end{solution}

\Exercise3 Let $V$ be the space of all $n\times n$ matrices over a
field $F$ and let $B$ be a fixed $n\times n$ matrix. If $T$ is the
linear operator on $V$ defined by $T(A) = AB - BA$, and if $f$ is the
trace function, what is $T^tf$?
\begin{solution}
  From Exercise~\ref{exercise:lin-tran:trace-AB-eq-trace-BA}, we know
  that $f(AB) = f(BA)$. So,
  \begin{align*}
    (T^tf)(A)
    &= f(TA) \\
    &= f(AB - BA) \\
    &= f(AB) - f(BA) \\
    &= 0,
  \end{align*}
  and we see that $T^tf = 0$.
\end{solution}
